\section{Purpose}
\label{sec:purpose}%
The InternHub - Students \& Companies (S\&C) platform requires this Installation and Technical Document (ITD) in order to be set up and run. It has been painstakingly created to walk administrators, developers, and stakeholders through each stage of installation and upkeep. This paper guarantees a smooth setup process, reducing potential problems and promoting effective deployment by offering precise and comprehensive instructions.
The ITD provides a comprehensive explanation of the platform's function, technical specifications, and architectural design in addition to installation. It highlights how crucial it is to manage dependencies and environment configurations appropriately in order to ensure seamless functioning across several systems. With this guide, users may deploy and configure the S&C platform with confidence, knowing that they have a trustworthy resource to handle any issues that may come up.

Fundamentally, the goal of this paper is to make the platform configuration less complicated so that users with different degrees of technical expertise can utilize it. This ITD guarantees that you have the skills and resources required to succeed, whether you're a developer working on the backend, an administrator managing the deployment, or a stakeholder trying to comprehend the operation of the system. It lays the groundwork for a strong and effective internship management system that connects students, businesses, and educational institutions by promoting clarity and minimizing uncertainty.Fundamentally, the goal of this paper is to make the platform configuration less complicated so that users with different degrees of technical expertise can utilize it. This ITD guarantees that you have the skills and resources required to succeed, whether you're a developer working on the backend, an administrator managing the deployment, or a stakeholder trying to comprehend the operation of the system. It lays the groundwork for a strong and effective internship management system that connects students, businesses, and educational institutions by promoting clarity and minimizing uncertainty.


\section{Scope}
\label{subsec:scope}%
\newcounter{g}
\setcounter{g}{1}
\newcommand{\cg}{\theg\stepcounter{g}}

By connecting students, businesses, and academic institutions, \textbf{InternHub - Students \& Companies (S\&C)} is a platform that aims to improve and expedite the internship experience. It seeks to match students with appropriate internships, allowing universities to manage the full internship lifecycle and businesses to identify the best candidates.

Companies may post extensive internship openings, analyze applications, and oversee the selection process on the site, while students can create detailed profiles, apply for internships, and follow their progress. Universities serve as supervisors, guaranteeing adherence to academic standards and offering systems for observation and criticism.

The platform aims to streamline processes and promote openness and cooperation among stakeholders with features like intelligent recommendations, feedback and quality assurance tools, and strong communication channels. In order to guarantee a smooth, equitable, and effective experience for every user, it also incorporates analytics, training materials, and a grievance management system.


\section{Definitions, Acronyms, Abbreviations}
\label{subsec:goals}%
\newcounter{g}
\setcounter{g}{1}
\newcommand{\cg}{\theg\stepcounter{g}}

Below is a table that lists all the goals of the S\&C platform:

\begin{table}[h!]
    \centering
    \begin{tabular}{|c|l|}
        \hline
        \textbf{Term/Acronym} & \textbf{Definition} \\ \hline
        S\&C & Students \& Companies Platform \\ \hline
        RASD & Requirements Analysis \& Specification Document \\ \hline
        CV & Curriculum Vitae \\ \hline
        UI & User Interface \\ \hline
        API & Application Programming Interface \\ \hline
        DBMS & Database Management System \\ \hline
        SLA & Service Level Agreement \\ \hline
        GDPR & General Data Protection Regulation \\ \hline
        ITD & Installation and Technical Document \\ \hline
    \end{tabular}
    \caption{Definitions, Acronyms, and Abbreviations}
    \label{tab:definitions}
\end{table}

\section{Revision History}
\label{sec:revision_history}

\begin{table}[H]
    \centering
    \begin{tabular}{ |p{0.1\linewidth}|p{0.15\linewidth}|p{0.45\linewidth}|p{0.2\linewidth}| }
        \hline
        \textbf{Version} & \textbf{Date} & \textbf{Description} & \textbf{Authors} \\
        \hline
        1.0 & 03 January 2025 & Initial Release & 
        Shreesh Kumar Jha, \newline
        Samarth Bhatia \\
        \hline
        2.0 & 04 January 2025 & LaTeX Format and Minor Fixes & 
        Shreesh Kumar Jha, \newline
        Samarth Bhatia \\
        \hline
    \end{tabular}
    \caption{Revision History}
    \label{tab:revision_history}
\end{table}

\section{Reference Documents}
\label{sec:reference_documents}%
\begin{itemize}
    \item Reference to Previous Year Student Projects for Structuring the Document
    \item \href{https://github.com/JhaBhatiaSharma/JhaBhatiaSharma/tree/main/RASD\%20Doc}{\textbf{\textcolor{blue}{\underline{Specification Document Assignment}}}}
\end{itemize}

\section{Document Structure}
\label{sec:document_structure}%
The document is divided into the following sections, each focusing on specific aspects of the installation and technical setup:

\subsection*{Introduction}
\begin{itemize}
    \item Establishes the purpose, scope, and structure of the document.
    \item Defines the key acronyms, abbreviations, and references used throughout.
    \item Provides historical revisions and context for updates.
\end{itemize}

\subsection*{Implemented Requirements}
\begin{itemize}
    \item Lists the core functional and technical requirements fulfilled by the platform.
    \item Includes features such as student and company registration, internship management, and data handling.
\end{itemize}

\subsection*{Design Choices}
\begin{itemize}
    \item Describes the architectural patterns, frameworks, and libraries employed in the platform's development.
    \item Covers key decisions related to scalability, reliability, and maintainability.
\end{itemize}

\subsection*{Source Code Structure}
\begin{itemize}
    \item Provides a breakdown of the backend, frontend, and database components.
    \item Includes directory hierarchies, file purposes, and dependencies.
\end{itemize}

\subsection*{Testing}
\begin{itemize}
    \item Outlines testing methodologies such as unit, integration, and end-to-end tests.
    \item Describes tools and strategies used to validate the platform's functionality and performance.
\end{itemize}

\subsection*{Installation Instructions}
\begin{itemize}
    \item Offers step-by-step guidance for setting up the backend, frontend, and database.
    \item Details prerequisites, installation commands, and troubleshooting tips.
\end{itemize}

\subsection*{Effort Spent}
\begin{itemize}
    \item Summarizes the contributions and hours worked by each team member on the document and platform.
\end{itemize}

\subsection*{References}
\begin{itemize}
    \item Lists all the sources and references used in the document.
\end{itemize}
