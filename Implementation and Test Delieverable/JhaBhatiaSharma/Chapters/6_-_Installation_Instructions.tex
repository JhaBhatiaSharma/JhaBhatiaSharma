\section{Setup Guide}

\subsection{Clone the Repository}
Clone the repository using the following command:
\begin{verbatim}
git clone https://github.com/JhaBhatiaSharma/JhaBhatiaSharma.git
\end{verbatim}

Navigate to the project directory:
\begin{verbatim}
cd JhaBhatiaSharma
\end{verbatim}

\subsection{Environment Configuration}
Create a \texttt{.env} file in the root directory of the project.  
Add the following variables to the file:

\begin{verbatim}
PORT=5001
JWT_SECRET=your_jwt_secret_here
MONGO_URI=your_mongo_uri_here
\end{verbatim}

\subsection{Backend Setup}

\subsubsection{Requirements}
\begin{itemize}
    \item Internet connection
    \item At least 50MB free disk space for Node.js binary.
    \item At least 100MB free disk space for dependencies.
\end{itemize}

\subsubsection{Setup}
Navigate to the backend folder:
\begin{verbatim}
cd backend
\end{verbatim}

Install dependencies:
\begin{verbatim}
npm install
\end{verbatim}

\subsubsection{Run}
Start the backend server:
\begin{verbatim}
npx nodemon server.js
\end{verbatim}

\subsection{Frontend Setup}

\subsubsection{Requirements}
\begin{itemize}
    \item Internet connection
\end{itemize}

\subsubsection{Setup}
Navigate to the frontend folder:
\begin{verbatim}
cd frontend
\end{verbatim}

Install dependencies:
\begin{verbatim}
npm install
\end{verbatim}

\subsubsection{Run}
Start the frontend development server:
\begin{verbatim}
npm run dev
\end{verbatim}

\subsection{Easier Installation (Using Docker)}
For an easier installation, you can use **Docker** to set up both the frontend and backend automatically.

\begin{enumerate}
    \item Go to the project home directory:
    \begin{verbatim}
    cd JhaBhatiaSharma
    \end{verbatim}

    \item Run the following command to build and start the application:
    \begin{verbatim}
    docker-compose up --build
    \end{verbatim}

    \item Once the containers are running, access the application at:
    \begin{verbatim}
    http://localhost:5173/
    \end{verbatim}
\end{enumerate}

\subsection{Access the Backend}
Open your browser and navigate to the URL:
\begin{verbatim}
http://localhost:5001
\end{verbatim}



