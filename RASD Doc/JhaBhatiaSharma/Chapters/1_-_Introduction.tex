Internships form a crucial bridge between academic learning and real-world professional experience. However, many existing tools struggle to efficiently connect students with suitable internship opportunities, resulting in mismatches and administrative burdens for universities and companies alike. \textbf{InternHub – Students \& Companies (S\&C)} addresses this gap by unifying the entire internship cycle—from finding the right match to ensuring quality, accountability, and continuous improvement—within a single, integrated platform.

This \textbf{Requirements Analysis and Specification Document (RASD)} outlines the platform’s technical and functional specifications, serving both as a development roadmap and a contractual reference for stakeholders. By employing advanced matching algorithms, providing robust feedback mechanisms, and streamlining internship workflows, the S\&C platform ensures a seamless experience that benefits students, companies, and universities. Through this document, each party gains a clear understanding of the platform’s capabilities and the value it delivers, setting the foundation for more efficient and effective internships.

\newpage

\section{Purpose}
\label{sec:purpose}%
This Requirements Analysis and Specification Document (RASD) provides a comprehensive overview of the \textbf{InternHub - Students \& Companies (S\&C) platform}. Its primary purpose is to serve as both a guide for developers responsible for implementing the system specifications and as a contractual reference point for clients and contractors. Additionally, it offers a clear, precise, and unambiguous explanation of the platform’s features and limitations, empowering students, companies, and academic institutions to confirm that the system meets their needs and requirements.
The S\&C platform’s overarching goal is to transform how university students connect with companies for internships. To this end, it focuses on:
\begin{enumerate}
    \item Establishing an efficient system that matches students with suitable internship opportunities.
    \item Streamlining the entire internship lifecycle—from application through completion—to simplify both student and company workflows. 
    \item Utilizing smart recommendation algorithms to align student skills with company requirements, ensuring more accurate and beneficial matches.
    \item Providing robust monitoring and feedback tools to enhance transparency, accountability, and continuous improvement.
    \item Ensuring effective complaint management and maintaining high-quality standards throughout the internship process.
\end{enumerate}

By offering a seamless and impactful experience, the S\&C platform aims to serve as a trusted solution that addresses the requirements of all stakeholders—students, businesses, and universities—thereby ensuring a more efficient, productive, and rewarding internship ecosystem.


\subsection{Goals}
\label{subsec:goals}%
\newcounter{g}
\setcounter{g}{1}
\newcommand{\cg}{\theg\stepcounter{g}}

Below is a table that lists all the goals of the S\&C platform:

\begin{center}
    \begin{longtable}{ |l|p{0.9\linewidth}| }
        \hline
        \textbf{ID} & \textbf{Description} \\
        \hline
        G\cg & Enable students to create detailed profiles, including their CVs, skills, academic achievements, and interests. \\
        \hline
        G\cg & Allow companies to post comprehensive internship opportunities, detailing roles, requirements, benefits, and timelines. \\
        \hline
        G\cg & Provide intelligent recommendations that align student skills and preferences with internship opportunities. \\
        \hline
        G\cg & Equip universities with tools to effectively monitor, manage, and track student internship progress and performance. \\
        \hline
        G\cg & Implement feedback and rating systems to promote accountability and continuous improvement for both students and companies. \\
        \hline
        G\cg & Facilitate seamless communication between students, companies, and universities for better collaboration and coordination. \\
        \hline
        G\cg & Integrate a secure document management system for handling internship-related paperwork, such as contracts and certificates. \\
        \hline
        G\cg & Offer analytics and reporting tools to provide insights into internship trends, success rates, and areas for improvement. \\
        \hline
        G\cg & Support multilingual functionality to ensure accessibility for a diverse user base across regions. \\
        \hline
        G\cg & Implement a grievance redressal mechanism to resolve disputes and ensure fair treatment for all users. \\
        \hline
        G\cg & Provide training modules or resources to prepare students for internships, such as interview tips and skill-building exercises. \\
        \hline
        \caption{Goals table.}
        \label{tab:goals_tab}%
    \end{longtable}
\end{center}

\section{Scope}
\label{sec:scope}%
\subsection{World Phenomena}
\label{subsec:world_phenomena}%
\newcounter{wp}
\setcounter{wp}{1}
\newcommand{\cwp}{\thewp\stepcounter{wp}}
\textbf{World Phenomena:} 
The project addresses core operational areas that shape the interactions and processes among students, companies, and universities. These include:
\begin{enumerate}
    \item \textbf{Student Internship Process:} Students search for and apply to internships, creating personal profiles and uploading their CVs.
    \begin{itemize}
        \item ID: WP1 Students create profiles and upload their CVs.
        \item ID: WP4 Students apply for internships.
    \end{itemize}

    \item \textbf{Company Internship Management:} Companies post internship opportunities, define requirements, and review applications to select suitable candidates.
    \begin{itemize}
        \item ID: WP2 Companies post internship opportunities.
        \item ID: WP5 Companies review applications and select candidates.
    \end{itemize}

    \item \textbf{University Oversight:} Universities maintain a supervisory role, monitoring and managing internship activities to ensure quality and compliance.
    \begin{itemize}
        \item ID: WP3 Universities monitor internship progress.
    \end{itemize}

    \item \textbf{Interview Coordination and Selection:} The platform facilitates the scheduling and management of interviews and other selection procedures to ensure a smooth hiring process.

    \item \textbf{Feedback and Quality Assurance:} Feedback is collected from students, companies, and universities to ensure continuous improvement, enhanced user experience, and adherence to quality standards.
    \begin{itemize}
        \item ID: WP6 Feedback is collected from students and companies.
    \end{itemize}

    \item \textbf{Complaint Handling:} A robust complaint-handling system ensures issues are addressed efficiently, maintaining trust and transparency in all interactions.

    \item \textbf{Fair and Transparent Interactions:} The system is designed to foster an environment of fairness, transparency, and accountability among students, companies, and universities.
\end{enumerate}
\newpage

\subsection{Shared phenomena}
\label{subsec:shared_phenomena}%
\newcounter{sp}
\setcounter{sp}{1}
\newcommand{\csp} {\thesp\stepcounter{sp}}
\textbf{Shared Phenomena:}
The platform provides a suite of shared functionalities, ensuring seamless interaction and data exchange among all stakeholders (Students, Companies, and Universities):
\begin{enumerate}
    \item \textbf{User Account Management and Profiles:}
    \begin{itemize}
        \item \textbf{ID:} SP1 Students create accounts on the platform (Controller: Student, Observer: Platform).
        \item \textbf{ID:} SP2 Companies create accounts on the platform (Controller: Company, Observer: Platform).
        \item \textbf{ID:} SP3 Universities create accounts on the platform (Controller: University, Observer: Platform).
    \end{itemize}

\item Students maintain detailed profiles, including CVs, skills, and achievements, while companies and universities manage their respective institutional profiles.
    
\item \textbf{Internship Postings and Applications:}
Students can apply directly to posted internships, and companies can review and manage these applications.
\begin{itemize}
    \item \textbf{ID:} SP4 Students apply for internships (Controller: Student, Observer: Platform).
    \item \textbf{ID:} SP5 Companies review and manage applications (Controller: Company, Observer: Platform).
    \item \textbf{ID:} SP6 Universities track student applications (Controller: University, Observer: Platform).
\end{itemize}

\item \textbf{Feedback and Rating System:}
A comprehensive feedback mechanism enables stakeholders to exchange feedback, ratings, and reviews to ensure accountability and continuous improvement.
\begin{itemize}
    \item \textbf{ID:} SP7 Feedback is exchanged between stakeholders (Controller: All, Observer: Platform).
\end{itemize}

\item \textbf{Interview Scheduling and Notifications:}
Interviews are coordinated efficiently, with automated reminders and updates for both students and companies.

\item \textbf{Communication and Support Tools:}
The platform offers seamless communication channels for queries, updates, and issue resolution among students, companies, and universities.
\item \textbf{Document Management:}
Secure storage and sharing capabilities for documents, such as internship agreements, certificates, and other relevant files, ensure easy access and proper record-keeping.

\item \textbf{Analytics and Insights:}
Real-time analytics and reporting tools help all parties make informed decisions, monitor internship progress, and assess performance metrics.

\item \textbf{Multilingual Support:}
The platform supports multiple languages, accommodating a diverse global user base.

\item \textbf{Training and Preparation Resources:}
Students have access to resources like training materials and interview preparation tools, aiding them in securing and succeeding in internships.
\end{enumerate}


\section{Definition, Acronyms, Abbreviations}
\label{sec:definition_acronyms_abbreviations}%
\begin{table}[H]
    \centering
    \begin{tabular}{ |l|p{0.7\linewidth}| }
        \hline
        \textbf{Term/Acronym} & \textbf{Definition} \\
        \hline
        S\&C  & Students \& Companies Platform \\
        \hline
        RASD  & Requirements Analysis \& Specification Document \\
        \hline
        CV    & Curriculum Vitae \\
        \hline
        UI    & User Interface \\
        \hline
        API   & Application Programming Interface \\
        \hline
        DBMS  & Database Management System \\
        \hline
        SLA   & Service Level Agreement \\
        \hline
        GDPR  & General Data Protection Regulation \\
        \hline
    \end{tabular}
    \caption{Acronyms and terms used in the document.}
    \label{tab:acronyms_sc}
\end{table}

\section{Revision History}
\label{sec:revision_history}

\begin{table}[H]
    \centering
    \begin{tabular}{ |p{0.1\linewidth}|p{0.15\linewidth}|p{0.45\linewidth}|p{0.2\linewidth}| }
        \hline
        \textbf{Version} & \textbf{Date} & \textbf{Description} & \textbf{Authors} \\
        \hline
        0.1 & 8 December 2024 & Initial Release & 
        Shreesh Kumar Jha, \newline
        Samarth Bhatia, \newline
        Satvik Sharma \\
        \hline
        1.0 & 17 December 2024 & Structure Fix and Added Use Cases & 
        Shreesh Kumar Jha, \newline
        Samarth Bhatia \\
        \hline
        2.0 & 19 December 2024 & Alloy Modelling & 
        Shreesh Kumar Jha, \newline
        Samarth Bhatia \\
        \hline
        3.0 & 21 December 2024 & Final Version & 
        Shreesh Kumar Jha, \newline
        Samarth Bhatia \newline
        Satvik Sharma\\
        \hline
    \end{tabular}
    \caption{Revision History}
    \label{tab:revision_history}
\end{table}


\section{Reference Documents}
\label{sec:reference_documents}%
\begin{itemize}
    \item Reference to Previous Year Student Projects for Structuring the Document
    \item Specification Document Assignment
    \item IEEE Standard Documentation For RASD 
\end{itemize}

\newpage

\section{Document Structure}
\label{sec:document_structure}%
As shown below, the document is organized into six sections, each with a distinct focus:

\paragraph{Introduction:} The project's goals, purpose, and a succinct analysis of common and worldwide occurrences are presented in the introduction, containing acronyms and definitions to help you grasp the problem domain.

\paragraph{Overall Description:} Provides a thorough rundown of the issue, potential domains, and features of the product. explains limitations, dependencies, and assumptions as well.

\paragraph{Specific Requirements:} Provides a detailed description of both functional and non-functional needs, including those pertaining to external interfaces.

\paragraph{Formal Analysis Using Alloy:} Provides assertions and checks to validate the model outlined in previous parts.

\paragraph{Effort Spent:} Describes how each team member contributed to the writing of this paper.

\paragraph{References:} Provides a list of all the supplementary materials and references that were utilized to produce the document.