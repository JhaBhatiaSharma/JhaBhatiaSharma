\documentclass{ConfigurationFiles/Politecnico_Di_Milano}


% CONFIGURATIONS
\usepackage{parskip} % For 
\usepackage{booktabs}
\usepackage{xcolor}
%paragraph layout
\usepackage{setspace} % For using single or double spacing
\usepackage{emptypage} % To insert empty pages
\usepackage{multicol} % To write in multiple columns (executive summary)
\setlength\columnsep{15pt} % Column separation in executive summary
\setlength\parindent{0pt} % Indentation
\raggedbottom
\usepackage{multirow}

% PACKAGES FOR TITLES
\usepackage{titlesec}
% \titlespacing{\section}{left spacing}{before spacing}{after spacing}
\titlespacing{\section}{0pt}{3.3ex}{2ex}
\titlespacing{\subsection}{0pt}{3.3ex}{1.65ex}
\titlespacing{\subsubsection}{0pt}{3.3ex}{1ex}
\usepackage{color}
\usepackage{enumitem}


\usepackage{longtable}            % For multipage tables
\usepackage{array}                % For p{width} column types
\usepackage{booktabs}             % (Optional) for nicer rules
\usepackage{lmodern}              %
\usepackage{pdflscape}
% PACKAGES FOR LANGUAGE AND FONT
\usepackage[english]{babel} % The document is in English  
\usepackage[utf8]{inputenc} % UTF8 encoding
\usepackage[T1]{fontenc} % Font encoding
\usepackage[11pt]{moresize} % Big fonts

% PACKAGES FOR IMAGES
\usepackage{graphicx}
\usepackage{transparent} % Enables transparent images
\usepackage{eso-pic} % For the background picture on the title page
\usepackage{subfig} % Numbered and caption subfigures using \subfloat.
\usepackage{tikz} % A package for high-quality hand-made figures.
\usetikzlibrary{}
\graphicspath{{./images/}} % Directory of the images
\usepackage{amsthm} % Coloured "Theorem"
\usepackage{thmtools}
\usepackage{xcolor}
\usepackage{float}

% STANDARD MATH PACKAGES
\usepackage{amsmath}
\usepackage{amssymb}
\usepackage{amsfonts}
\usepackage{bm}
\usepackage[overload]{empheq} % For braced-style systems of equations.
\usepackage{fix-cm} % To override original LaTeX restrictions on sizes

% PACKAGES FOR TABLES
\usepackage{tabularx}
\usepackage{longtable} % Tables that can span several pages
\usepackage{colortbl}

% PACKAGES FOR ALGORITHMS (PSEUDO-CODE)
\usepackage{algorithm}
\usepackage{algorithmic}

\usepackage{pifont}

% PACKAGES FOR REFERENCES & BIBLIOGRAPHY
\usepackage[
    colorlinks=true,
    linkcolor=black,
    anchorcolor=black,
    citecolor=black,
    filecolor=black,
    menucolor=black,
    runcolor=black,
    urlcolor=black
]{hyperref} % Adds clickable links at references
\usepackage{cleveref}
\usepackage[square, numbers, sort&compress]{natbib} % Square brackets, citing references with numbers, citations sorted by appearance in the text and compressed
\bibliographystyle{abbrvnat} % You may use a different style adapted to your field

% OTHER PACKAGES
\usepackage{pdfpages} % To include a pdf file
\usepackage{afterpage}
\usepackage{lipsum} % DUMMY PACKAGE
\usepackage{fancyhdr}
\usepackage{wasysym} % For the headers
\usepackage{rotating}
\usepackage{listings}
\usepackage{hyperref}
\input{ConfigurationFiles/alloy.sty}
\fancyhf{}

% Input of configuration file. Do not change config.tex file unless you really know what you are doing. 
\input{ConfigurationFiles/config}


\definecolor{dkgreen}{rgb}{0,0.6,0}
\definecolor{gray}{rgb}{0.5,0.5,0.5}
\definecolor{mauve}{rgb}{0.58,0,0.82}

\lstset{frame=tb,
    language=alloy,
    aboveskip=3mm,
    belowskip=3mm,
    showstringspaces=false,
    columns=flexible,
    basicstyle={\small\ttfamily},
    numbers=none,
    numberstyle=\tiny\color{gray},
    keywordstyle=\bf\color{blue},
    commentstyle=\it\color{dkgreen},
    stringstyle=\color{mauve},
    breaklines=true,
    breakatwhitespace=true,
    tabsize=3
}







%----------------------------------------------------------------------------
%	BEGIN OF YOUR DOCUMENT
%----------------------------------------------------------------------------



\begin{document}
    \fancypagestyle{plain}{%
        \fancyhf{} % Clear all header and footer fields
        \fancyhead[RO,RE]{\thepage} %RO=right odd, RE=right even
        \renewcommand{\headrulewidth}{0pt}
        \renewcommand{\footrulewidth}{0pt}}

        
    \pagestyle{empty} % No page numbers
    \frontmatter % Use roman page numbering style (i, ii, iii, iv...) for the preamble pages

    \puttitle{
        title=Software Engineering 2\\Acceptance Testing Documentation,
        name1=Shreesh Kumar Jha - 11022306, % Author Name and Surname
        name2=Samarth Bhatia - 11059097,
        name3=Satvik Sharma - 11054680,
        academicyear=2024-2025,
        version=2.0,
        releasedate=6/02/2025,
    }
    
    
    \startpreamble
    \setcounter{page}{1} % Set page counter to 1


% TABLE OF CONTENTS
    \thispagestyle{empty}
    \tableofcontents % Table of contents
    \thispagestyle{empty}
    \cleardoublepage

    
    \addtocontents{toc}{\vspace{1em}} % Add a gap in the Contents, for aesthetics
    \mainmatter % Begin numeric (1,2,3...) page numbering


    \chapter{Introduction}
    \label{ch:introduction}%
    Internships form a crucial bridge between academic learning and real-world professional experience. However, many existing tools struggle to efficiently connect students with suitable internship opportunities, resulting in mismatches and administrative burdens for universities and companies alike. \textbf{InternHub – Students \& Companies (S\&C)} addresses this gap by unifying the entire internship cycle—from finding the right match to ensuring quality, accountability, and continuous improvement—within a single, integrated platform.

This \textbf{Requirements Analysis and Specification Document (RASD)} outlines the platform’s technical and functional specifications, serving both as a development roadmap and a contractual reference for stakeholders. By employing advanced matching algorithms, providing robust feedback mechanisms, and streamlining internship workflows, the S\&C platform ensures a seamless experience that benefits students, companies, and universities. Through this document, each party gains a clear understanding of the platform’s capabilities and the value it delivers, setting the foundation for more efficient and effective internships.

\newpage

\section{Purpose}
\label{sec:purpose}%
This Requirements Analysis and Specification Document (RASD) provides a comprehensive overview of the \textbf{InternHub - Students \& Companies (S\&C) platform}. Its primary purpose is to serve as both a guide for developers responsible for implementing the system specifications and as a contractual reference point for clients and contractors. Additionally, it offers a clear, precise, and unambiguous explanation of the platform’s features and limitations, empowering students, companies, and academic institutions to confirm that the system meets their needs and requirements.
The S\&C platform’s overarching goal is to transform how university students connect with companies for internships. To this end, it focuses on:
\begin{enumerate}
    \item Establishing an efficient system that matches students with suitable internship opportunities.
    \item Streamlining the entire internship lifecycle—from application through completion—to simplify both student and company workflows. 
    \item Utilizing smart recommendation algorithms to align student skills with company requirements, ensuring more accurate and beneficial matches.
    \item Providing robust monitoring and feedback tools to enhance transparency, accountability, and continuous improvement.
    \item Ensuring effective complaint management and maintaining high-quality standards throughout the internship process.
\end{enumerate}

By offering a seamless and impactful experience, the S\&C platform aims to serve as a trusted solution that addresses the requirements of all stakeholders—students, businesses, and universities—thereby ensuring a more efficient, productive, and rewarding internship ecosystem.


\subsection{Goals}
\label{subsec:goals}%
\newcounter{g}
\setcounter{g}{1}
\newcommand{\cg}{\theg\stepcounter{g}}

Below is a table that lists all the goals of the S\&C platform:

\begin{center}
    \begin{longtable}{ |l|p{0.9\linewidth}| }
        \hline
        \textbf{ID} & \textbf{Description} \\
        \hline
        G\cg & Enable students to create detailed profiles, including their CVs, skills, academic achievements, and interests. \\
        \hline
        G\cg & Allow companies to post comprehensive internship opportunities, detailing roles, requirements, benefits, and timelines. \\
        \hline
        G\cg & Provide intelligent recommendations that align student skills and preferences with internship opportunities. \\
        \hline
        G\cg & Equip universities with tools to effectively monitor, manage, and track student internship progress and performance. \\
        \hline
        G\cg & Implement feedback and rating systems to promote accountability and continuous improvement for both students and companies. \\
        \hline
        G\cg & Facilitate seamless communication between students, companies, and universities for better collaboration and coordination. \\
        \hline
        G\cg & Integrate a secure document management system for handling internship-related paperwork, such as contracts and certificates. \\
        \hline
        G\cg & Offer analytics and reporting tools to provide insights into internship trends, success rates, and areas for improvement. \\
        \hline
        G\cg & Support multilingual functionality to ensure accessibility for a diverse user base across regions. \\
        \hline
        G\cg & Implement a grievance redressal mechanism to resolve disputes and ensure fair treatment for all users. \\
        \hline
        G\cg & Provide training modules or resources to prepare students for internships, such as interview tips and skill-building exercises. \\
        \hline
        \caption{Goals table.}
        \label{tab:goals_tab}%
    \end{longtable}
\end{center}

\section{Scope}
\label{sec:scope}%
\subsection{World Phenomena}
\label{subsec:world_phenomena}%
\newcounter{wp}
\setcounter{wp}{1}
\newcommand{\cwp}{\thewp\stepcounter{wp}}
\textbf{World Phenomena:} 
The project addresses core operational areas that shape the interactions and processes among students, companies, and universities. These include:
\begin{enumerate}
    \item \textbf{Student Internship Process:} Students search for and apply to internships, creating personal profiles and uploading their CVs.
    \begin{itemize}
        \item ID: WP1 Students create profiles and upload their CVs.
        \item ID: WP4 Students apply for internships.
    \end{itemize}

    \item \textbf{Company Internship Management:} Companies post internship opportunities, define requirements, and review applications to select suitable candidates.
    \begin{itemize}
        \item ID: WP2 Companies post internship opportunities.
        \item ID: WP5 Companies review applications and select candidates.
    \end{itemize}

    \item \textbf{University Oversight:} Universities maintain a supervisory role, monitoring and managing internship activities to ensure quality and compliance.
    \begin{itemize}
        \item ID: WP3 Universities monitor internship progress.
    \end{itemize}

    \item \textbf{Interview Coordination and Selection:} The platform facilitates the scheduling and management of interviews and other selection procedures to ensure a smooth hiring process.

    \item \textbf{Feedback and Quality Assurance:} Feedback is collected from students, companies, and universities to ensure continuous improvement, enhanced user experience, and adherence to quality standards.
    \begin{itemize}
        \item ID: WP6 Feedback is collected from students and companies.
    \end{itemize}

    \item \textbf{Complaint Handling:} A robust complaint-handling system ensures issues are addressed efficiently, maintaining trust and transparency in all interactions.

    \item \textbf{Fair and Transparent Interactions:} The system is designed to foster an environment of fairness, transparency, and accountability among students, companies, and universities.
\end{enumerate}
\newpage

\subsection{Shared phenomena}
\label{subsec:shared_phenomena}%
\newcounter{sp}
\setcounter{sp}{1}
\newcommand{\csp} {\thesp\stepcounter{sp}}
\textbf{Shared Phenomena:}
The platform provides a suite of shared functionalities, ensuring seamless interaction and data exchange among all stakeholders (Students, Companies, and Universities):
\begin{enumerate}
    \item \textbf{User Account Management and Profiles:}
    \begin{itemize}
        \item \textbf{ID:} SP1 Students create accounts on the platform (Controller: Student, Observer: Platform).
        \item \textbf{ID:} SP2 Companies create accounts on the platform (Controller: Company, Observer: Platform).
        \item \textbf{ID:} SP3 Universities create accounts on the platform (Controller: University, Observer: Platform).
    \end{itemize}

\item Students maintain detailed profiles, including CVs, skills, and achievements, while companies and universities manage their respective institutional profiles.
    
\item \textbf{Internship Postings and Applications:}
Students can apply directly to posted internships, and companies can review and manage these applications.
\begin{itemize}
    \item \textbf{ID:} SP4 Students apply for internships (Controller: Student, Observer: Platform).
    \item \textbf{ID:} SP5 Companies review and manage applications (Controller: Company, Observer: Platform).
    \item \textbf{ID:} SP6 Universities track student applications (Controller: University, Observer: Platform).
\end{itemize}

\item \textbf{Feedback and Rating System:}
A comprehensive feedback mechanism enables stakeholders to exchange feedback, ratings, and reviews to ensure accountability and continuous improvement.
\begin{itemize}
    \item \textbf{ID:} SP7 Feedback is exchanged between stakeholders (Controller: All, Observer: Platform).
\end{itemize}

\item \textbf{Interview Scheduling and Notifications:}
Interviews are coordinated efficiently, with automated reminders and updates for both students and companies.

\item \textbf{Communication and Support Tools:}
The platform offers seamless communication channels for queries, updates, and issue resolution among students, companies, and universities.
\item \textbf{Document Management:}
Secure storage and sharing capabilities for documents, such as internship agreements, certificates, and other relevant files, ensure easy access and proper record-keeping.

\item \textbf{Analytics and Insights:}
Real-time analytics and reporting tools help all parties make informed decisions, monitor internship progress, and assess performance metrics.

\item \textbf{Multilingual Support:}
The platform supports multiple languages, accommodating a diverse global user base.

\item \textbf{Training and Preparation Resources:}
Students have access to resources like training materials and interview preparation tools, aiding them in securing and succeeding in internships.
\end{enumerate}


\section{Definition, Acronyms, Abbreviations}
\label{sec:definition_acronyms_abbreviations}%
\begin{table}[H]
    \centering
    \begin{tabular}{ |l|p{0.7\linewidth}| }
        \hline
        \textbf{Term/Acronym} & \textbf{Definition} \\
        \hline
        S\&C  & Students \& Companies Platform \\
        \hline
        RASD  & Requirements Analysis \& Specification Document \\
        \hline
        CV    & Curriculum Vitae \\
        \hline
        UI    & User Interface \\
        \hline
        API   & Application Programming Interface \\
        \hline
        DBMS  & Database Management System \\
        \hline
        SLA   & Service Level Agreement \\
        \hline
        GDPR  & General Data Protection Regulation \\
        \hline
    \end{tabular}
    \caption{Acronyms and terms used in the document.}
    \label{tab:acronyms_sc}
\end{table}

\section{Revision History}
\label{sec:revision_history}

\begin{table}[H]
    \centering
    \begin{tabular}{ |p{0.1\linewidth}|p{0.15\linewidth}|p{0.45\linewidth}|p{0.2\linewidth}| }
        \hline
        \textbf{Version} & \textbf{Date} & \textbf{Description} & \textbf{Authors} \\
        \hline
        0.1 & 8 December 2024 & Initial Release & 
        Shreesh Kumar Jha, \newline
        Samarth Bhatia, \newline
        Satvik Sharma \\
        \hline
        1.0 & 17 December 2024 & Structure Fix and Added Use Cases & 
        Shreesh Kumar Jha, \newline
        Samarth Bhatia \\
        \hline
        2.0 & 19 December 2024 & Alloy Modelling & 
        Shreesh Kumar Jha, \newline
        Samarth Bhatia \\
        \hline
        3.0 & 21 December 2024 & Final Version & 
        Shreesh Kumar Jha, \newline
        Samarth Bhatia \newline
        Satvik Sharma\\
        \hline
    \end{tabular}
    \caption{Revision History}
    \label{tab:revision_history}
\end{table}


\section{Reference Documents}
\label{sec:reference_documents}%
\begin{itemize}
    \item Reference to Previous Year Student Projects for Structuring the Document
    \item Specification Document Assignment
    \item IEEE Standard Documentation For RASD 
\end{itemize}

\newpage

\section{Document Structure}
\label{sec:document_structure}%
As shown below, the document is organized into six sections, each with a distinct focus:

\paragraph{Introduction:} The project's goals, purpose, and a succinct analysis of common and worldwide occurrences are presented in the introduction, containing acronyms and definitions to help you grasp the problem domain.

\paragraph{Overall Description:} Provides a thorough rundown of the issue, potential domains, and features of the product. explains limitations, dependencies, and assumptions as well.

\paragraph{Specific Requirements:} Provides a detailed description of both functional and non-functional needs, including those pertaining to external interfaces.

\paragraph{Formal Analysis Using Alloy:} Provides assertions and checks to validate the model outlined in previous parts.

\paragraph{Effort Spent:} Describes how each team member contributed to the writing of this paper.

\paragraph{References:} Provides a list of all the supplementary materials and references that were utilized to produce the document.
    \chapter{Implemented Requirements}
    \label{ch:architectural_design}%
    \section{Overview}
\label{sec:overview}
We adhered to the installation instructions found in \textbf{IT Document (ITD)v1}. The installation procedure was simple because of the comprehensive and well-organized guide. We encountered no problems throughout the installation, in contrast to other intricate setups. Every stage went without a hitch, and the system was successfully and consistently deployed.

\section{Dependency Installation}
\subsection{Backend Dependencies:}
To install the backend, we followed these steps:

\subsubsection{Installed Python and Virtual Environment}
\begin{itemize}
    \item Created a virtual environment and activated it using:
\end{itemize}


\begin{verbatim}
python3 -m venv env
source env/bin/activate  # For Linux/Mac  
env\Scripts\activate  # For Windows  
\end{verbatim}

\subsubsection{Installed Required Python Packages}
\begin{itemize}
    \item Using the \texttt{requirements.txt} file from the repository:
\end{itemize}

\begin{verbatim}
pip install -r requirements.txt
\end{verbatim}

\subsubsection{Installed Node.js and npm (For API Calls Testing)}
\begin{itemize}
    \item Installed Node.js (latest LTS version) and verified npm was installed properly:
\end{itemize}

\begin{verbatim}
node -v
npm -v
\end{verbatim}

\subsubsection{Installed MongoDB as the Database}
\begin{itemize}
    \item Since the backend uses MongoDB, we installed and ran it using:
\end{itemize}

\begin{verbatim}
mongod --dbpath=/data/db
\end{verbatim}

\subsection{Frontend Dependencies}

The frontend is built using React. The following dependencies were installed:

\subsubsection{Installed React and Required Libraries}
\begin{verbatim}
npm install
\end{verbatim}

\subsubsection{Checked TailwindCSS and Other UI Libraries}
\begin{itemize}
    \item Verified TailwindCSS and Flowbite were installed correctly.
\end{itemize}

\subsection{Backend Dependencies}

To install the backend, we followed these steps:

\subsubsection{Installed Python and Virtual Environment}
\begin{itemize}
    \item Created a virtual environment and activated it using:
\end{itemize}

\begin{verbatim}
python3 -m venv env
source env/bin/activate  # For Linux/Mac  
env\Scripts\activate  # For Windows  
\end{verbatim}

\subsubsection{Installed Required Python Packages}
\begin{itemize}
    \item Using the \texttt{requirements.txt} file from the repository:
\end{itemize}

\begin{verbatim}
pip install -r requirements.txt
\end{verbatim}

\subsubsection{Installed Node.js and npm (For API Calls Testing)}
\begin{itemize}
    \item Installed Node.js (latest LTS version) and verified npm was installed properly:
\end{itemize}

\begin{verbatim}
node -v
npm -v
\end{verbatim}

\subsubsection{Installed MongoDB as the Database}
\begin{itemize}
    \item Since the backend uses MongoDB, we installed and ran it using:
\end{itemize}

\begin{verbatim}
mongod --dbpath=/data/db
\end{verbatim}

\subsection{Frontend Dependencies}

The frontend is built using React. The following dependencies were installed:

\subsubsection*{Installed React and Required Libraries}
\begin{verbatim}
npm install
\end{verbatim}

\subsubsection*{Checked TailwindCSS and Other UI Libraries}
\begin{itemize}
    \item Verified TailwindCSS and Flowbite were installed correctly.
\end{itemize}

\section{Backend Installation}

\subsection{Cloned the Repository}
\begin{verbatim}
git clone https://github.com/edogriba/GribaldoRosa.git
cd ITD/backend
\end{verbatim}

\subsection{Started the Backend Server}
Activated the virtual environment and ran the server:
\begin{verbatim}
source env/bin/activate  # For Linux/Mac
python3 run.py
\end{verbatim}

\subsection{Database Initialization}
Ran the \texttt{createDB.py} script to set up MongoDB collections.

\section{Frontend Installation}

\subsection{Navigated to the Frontend Directory}
\begin{verbatim}
cd ../frontend
\end{verbatim}

\subsection{Installed Frontend Dependencies}
\begin{verbatim}
npm install
\end{verbatim}

\subsection{Ran the Frontend Application}
\begin{verbatim}
npm start
\end{verbatim}

\subsection*{Accessed the Application on Localhost}
Opened \texttt{http://localhost:3000} in a browser.

\section*{Testing Environment Setup}

\subsection{Tested API Endpoints using Postman}
\begin{itemize}
    \item All API requests responded as expected.
\end{itemize}

\subsection*{Tested User Registration \& Login}
\begin{itemize}
    \item Successfully created and logged in as a Student, Recruiter, and Admin.
\end{itemize}

\subsection{Internship Search \& Applications}
\begin{itemize}
    \item Verified that students could search and apply for internships.
\end{itemize}

\subsection{Messaging \& Complaints System}
\begin{itemize}
    \item Messages and complaints were successfully created and retrieved.
\end{itemize}

\section{Problems Encountered \& Documentation Incoherences}
\begin{itemize}
    \item \textbf{No Issues Faced During Installation.}
    \item The provided IT Document (ITD) was clear and accurate, ensuring a smooth setup.
    \item There were no missing dependencies or undocumented configurations.
\end{itemize}

\section{Observations}

\begin{itemize}
    \item Well-structured documentation made installation easy.
    \item Database initialization was straightforward without requiring manual data imports.
    \item No additional configurations were needed beyond those in the ITD.
\end{itemize}



    \chapter{Acceptance Test Case}
    \label{ch:user_interface_design}%
    \section{Acceptance Test Cases}

The \textbf{Students \& Companies (S\&C) platform} requirements are divided into two main categories:

\begin{itemize}
    \item \textbf{Core Requirements}: Fundamental system features that must be implemented and functional.
    \item \textbf{Goal Reaching Requirements}: Additional functionalities mapped to user goals, improving usability and efficiency.
\end{itemize}

The following sections outline some important test cases, linking each requirement (R1, R2, etc.) from the \textit{RASD} document to its corresponding test.  
\textbf{Overall: 23 Backend Tests.}

\newpage
\subsection{Core Requirements (Users)}

\begin{table}[h]
    \centering
    \renewcommand{\arraystretch}{1.3}
    \begin{tabular}{|c|p{4cm}|p{4cm}|p{4cm}|c|}
        \hline
        \textbf{Requirement} & \textbf{Test Description} & \textbf{Expected Outcome} & \textbf{Actual Outcome} & \textbf{Status} \\
        \hline
        R1 & Allow users to sign up. & Users can register successfully. & \ding{51} Works as expected. & \ding{51} Pass \\
        \hline
        R2 & Allow users to fill in profile information when signing up. & Profile data saved successfully. & \ding{51} Works as expected. & \ding{51} Pass \\
        \hline
        R3 & Allow users to log in. & Users can access their accounts. & \ding{51} Works as expected. & \ding{51} Pass \\
        \hline
        R4 & Allow users to log out. & Users are logged out securely. & \ding{51} Works as expected. & \ding{51} Pass \\
        \hline
        R5 & Allow users to update their profile information. & Profile updates are saved. & \ding{51} Works as expected. & \ding{51} Pass \\
        \hline
        R6 & Allow users to examine their own internships. & Users can see the internships they applied for. & \ding{51} Works as expected. & \ding{51} Pass \\
        \hline
    \end{tabular}
    \caption{Core Requirements Test Cases for Users}
    \label{tab:core_requirements}
\end{table}

\newpage
\subsection{Student Requirements}

\begin{table}[h]
    \centering
    \renewcommand{\arraystretch}{1.3}
    \begin{tabular}{|c|p{3cm}|p{3cm}|p{3cm}|c|}
        \hline
        \textbf{Requirement} & \textbf{Test Description} & \textbf{Expected Outcome} & \textbf{Actual Outcome} & \textbf{Status} \\
        \hline
        R7 & Allow students to examine open internship positions. & Available internships are listed. & \ding{51} Works as expected. & \ding{51} Pass \\
        \hline
        R8 & Allow students to examine their own applications. & Applications and statuses are visible. & \ding{51} Works as expected. & \ding{51} Pass \\
        \hline
        R9 & Allow students to search for a specific internship position. & Search results match the criteria. & \ding{51} Works as expected. & \ding{51} Pass \\
        \hline
        R10 & Allow students to apply for an internship position. & Application is submitted successfully. & \ding{51} Works as expected. & \ding{51} Pass \\
        \hline
        R11 & Notify students when a suitable internship is opened. & Students receive notifications for relevant internships. & \ding{51} Works as expected. & \ding{51} Pass \\
        \hline
        R12 & List different internship positions aligned with student profiles. & Internships displayed match student skills. & \ding{51} Works as expected. & \ding{51} Pass \\
        \hline
    \end{tabular}
    \caption{Student Requirements Test Cases}
    \label{tab:student_requirements}
\end{table}

\newpage
\subsection{Application Requirements}

\begin{table}[h]
    \centering
    \renewcommand{\arraystretch}{1.3}
    \begin{tabular}{|c|p{4cm}|p{4cm}|p{4.5cm}|c|}
        \hline
        \textbf{Requirement} & \textbf{Test Description} & \textbf{Expected Outcome} & \textbf{Actual Outcome} & \textbf{Status} \\
        \hline
        R13 & Allow students to confirm or refuse an accepted internship offer. & Students can accept/reject offers. & \ding{51} Works as expected. & \ding{51} Pass \\
        \hline
        R14 & Allow companies to request skill assessments and schedule interviews. & Recruiters can request interviews. & \ding{51} Works as expected. & \ding{51} Pass \\
        \hline
        R15 & Allow students to access interview details and links. & Interview details are available in the dashboard. & \ding{51} Works as expected. & \ding{51} Pass \\
        \hline
        R16 & Allow students to see the status of their applications. & Application status updates correctly. & \ding{51} Works as expected. & \ding{51} Pass \\
        \hline
    \end{tabular}
    \caption{Application Requirements Test Cases}
    \label{tab:application_requirements}
\end{table}

\newpage
\subsection{Company Requirements}

\begin{table}[h]
    \centering
    \renewcommand{\arraystretch}{1.3}
    \begin{tabular}{|c|p{4cm}|p{3.5cm}|p{4cm}|c|}
        \hline
        \textbf{Requirement} & \textbf{Test Description} & \textbf{Expected Outcome} & \textbf{Actual Outcome} & \textbf{Status} \\
        \hline
        R17 & Allow companies to open and examine internship positions. & Companies can create and view internships. & \ding{51} Works as expected. & \ding{51} Pass \\
        \hline
        R18 & Allow companies to accept or reject applications. & Recruiters can update application statuses. & \ding{51} Works as expected. & \ding{51} Pass \\
        \hline
        R19 & Allow companies to close internship positions. & Internship status updates to "Closed." & \ding{51} Works as expected. & \ding{51} Pass \\
        \hline
        R20 & Notify companies when a new relevant student profile is available. & Recruiters receive notifications about matching students. & \ding{51} Works as expected. & \ding{51} Pass \\
        \hline
    \end{tabular}
    \caption{Company Requirements Test Cases}
    \label{tab:company_requirements}
\end{table}

\newpage
\subsection{Feedback and Suggestions}
\begin{table}[h]
    \centering
    \renewcommand{\arraystretch}{1.3}
    \begin{tabular}{|c|p{4.5cm}|p{3.5cm}|p{4.5cm}|c|}
        \hline
        \textbf{Requirement} & \textbf{Test Description} & \textbf{Expected Outcome} & \textbf{Actual Outcome} & \textbf{Status} \\
        \hline
        R24 & Allow students and companies to rate the internship experience. & Ratings are recorded successfully. & \ding{51} Works as expected. & \ding{51} Pass \\
        \hline
        R25 & Allow students to provide feedback on the internship experience. & Students can submit feedback. & \ding{51} Works as expected. & \ding{51} Pass \\
        \hline
        R26 & Allow companies to send feedback/news to students. & Students receive feedback notifications. & \ding{51} Works as expected. & \ding{51} Pass \\
        \hline
        R27 & Allow both parties to file complaints about the internship. & Complaints are submitted and reviewed. & \ding{51} Works as expected. & \ding{51} Pass \\
        \hline
    \end{tabular}
    \caption{Feedback and Suggestions Test Cases}
    \label{tab:feedback_suggestions}
\end{table}

\subsection{Database Tests}

The following tests verify the database operations for different modules in the \textbf{Students \& Companies (S\&C) platform}. These tests check insertion, retrieval, updates, and constraints for key entities.

\subsubsection{Assessment Database Tests}

\begin{table}[h]
    \centering
    \renewcommand{\arraystretch}{1.3}
    \begin{tabular}{|p{5cm}|p{5cm}|p{4cm}|c|}
        \hline
        \textbf{Test Case} & \textbf{Description} & \textbf{Expected Outcome} & \textbf{Status} \\
        \hline
        Insert assessment & Add a new assessment record & \ding{51} Successfully inserted & \ding{51} Pass \\
        \hline
        Invalid insert & Insert invalid assessment data & \ding{55} Raises exception & \ding{51} Pass \\
        \hline
        Retrieve last assessment by application ID & Fetch last added assessment for an application & \ding{51} Retrieves latest assessment & \ding{51} Pass \\
        \hline
        Retrieve with invalid application ID & Query non-existent application ID & \ding{55} Returns None & \ding{51} Pass \\
        \hline
    \end{tabular}
    \caption{Assessment Database Test Cases}
    \label{tab:assessment_database_tests}
\end{table}

\subsubsection{Application Database Tests}

\begin{table}[h]
    \centering
    \renewcommand{\arraystretch}{1.3}
    \begin{tabular}{|p{5cm}|p{5cm}|p{4cm}|c|}
        \hline
        \textbf{Test Case} & \textbf{Description} & \textbf{Expected Outcome} & \textbf{Status} \\
        \hline
        Insert application & Add a new application entry & \ding{51} Successfully inserted & \ding{51} Pass \\
        \hline
        Invalid insert & Insert application with missing fields & \ding{55} Raises exception & \ding{51} Pass \\
        \hline
        Retrieve application by ID & Fetch an application record & \ding{51} Retrieves correct data & \ding{51} Pass \\
        \hline
        Retrieve application by student ID & Get applications for a student & \ding{51} Retrieves correct data & \ding{51} Pass \\
        \hline
        Retrieve application by internship ID & Get applications for an internship & \ding{51} Retrieves correct data & \ding{51} Pass \\
        \hline
        Update application status & Modify an application status & \ding{51} Status updates successfully & \ding{51} Pass \\
        \hline
    \end{tabular}
    \caption{Application Database Test Cases}
    \label{tab:application_database_tests}
\end{table}

\subsubsection{Company Database Tests}

\begin{table}[h]
    \centering
    \renewcommand{\arraystretch}{1.3}
    \begin{tabular}{|p{5cm}|p{5cm}|p{4cm}|c|}
        \hline
        \textbf{Test Case} & \textbf{Description} & \textbf{Expected Outcome} & \textbf{Status} \\
        \hline
        Insert company & Add a new company entry & \ding{51} Successfully inserted & \ding{51} Pass \\
        \hline
        Duplicate company insert & Try inserting a duplicate company & \ding{55} Raises exception & \ding{51} Pass \\
        \hline
        Retrieve company by ID & Fetch a company record & \ding{51} Retrieves correct data & \ding{51} Pass \\
        \hline
        Retrieve company by email & Fetch company details using email & \ding{51} Retrieves correct data & \ding{51} Pass \\
        \hline
        Update company details & Modify company profile data & \ding{51} Updates successfully & \ding{51} Pass \\
        \hline
    \end{tabular}
    \caption{Company Database Test Cases}
    \label{tab:company_database_tests}
\end{table}

\newpage
\subsubsection{Complaint Database Tests}

\begin{table}[h]
    \centering
    \renewcommand{\arraystretch}{1.3}
    \begin{tabular}{|p{5cm}|p{5cm}|p{4cm}|c|}
        \hline
        \textbf{Test Case} & \textbf{Description} & \textbf{Expected Outcome} & \textbf{Status} \\
        \hline
        Insert complaint & Add a new complaint entry & \ding{51} Successfully inserted & \ding{51} Pass \\
        \hline
        Invalid insert & Insert complaint with missing fields & \ding{55} Raises exception & \ding{51} Pass \\
        \hline
        Retrieve complaints by internship ID & Fetch complaints for an internship & \ding{51} Retrieves correct data & \ding{51} Pass \\
        \hline
        Retrieve complaints for non-existent internship & Query complaints for a non-existent ID & \ding{55} Returns empty list & \ding{51} Pass \\
        \hline
    \end{tabular}
    \caption{Complaint Database Test Cases}
    \label{tab:complaint_database_tests}
\end{table}

\subsubsection{Internship Database Tests}

\begin{table}[h]
    \centering
    \renewcommand{\arraystretch}{1.3}
    \begin{tabular}{|p{5cm}|p{5cm}|p{4cm}|c|}
        \hline
        \textbf{Test Case} & \textbf{Description} & \textbf{Expected Outcome} & \textbf{Status} \\
        \hline
        Insert internship & Add a new internship entry & \ding{51} Successfully inserted & \ding{51} Pass \\
        \hline
        Invalid insert & Insert internship with missing fields & \ding{55} Raises exception & \ding{51} Pass \\
        \hline
        Retrieve internship by ID & Fetch internship details & \ding{51} Retrieves correct data & \ding{51} Pass \\
        \hline
        Retrieve internship by application ID & Fetch internship for a given application & \ding{51} Retrieves correct data & \ding{51} Pass \\
        \hline
        Retrieve internships by company ID & Fetch all internships posted by a company & \ding{51} Retrieves correct data & \ding{51} Pass \\
        \hline
        Update internship status & Modify internship state (e.g., ongoing, finished) & \ding{51} Updates successfully & \ding{51} Pass \\
        \hline
    \end{tabular}
    \caption{Internship Database Test Cases}
    \label{tab:internship_database_tests}
\end{table}

\newpage
\subsubsection{Internship Position Database Tests}

\begin{table}[h]
    \centering
    \renewcommand{\arraystretch}{1.3}
    \begin{tabular}{|p{5cm}|p{5cm}|p{4cm}|c|}
        \hline
        \textbf{Test Case} & \textbf{Description} & \textbf{Expected Outcome} & \textbf{Status} \\
        \hline
        Insert internship position & Add a new internship listing & \ding{51} Successfully inserted & \ding{51} Pass \\
        \hline
        Invalid insert & Insert internship position with missing fields & \ding{55} Raises exception & \ding{51} Pass \\
        \hline
        Retrieve by ID & Fetch internship position details & \ding{51} Retrieves correct data & \ding{51} Pass \\
        \hline
        Retrieve by company ID & Fetch all internship positions for a company & \ding{51} Retrieves correct data & \ding{51} Pass \\
        \hline
        Retrieve by program name & Fetch internship positions by program name & \ding{51} Retrieves correct data & \ding{51} Pass \\
        \hline
        Search with filters & Filter internships by role, location, stipend & \ding{51} Retrieves filtered results & \ding{51} Pass \\
        \hline
        Update internship position status & Modify internship position state (e.g., Open, Closed) & \ding{51} Updates successfully & \ding{51} Pass \\
        \hline
    \end{tabular}
    \caption{Internship Position Database Test Cases}
    \label{tab:internship_position_database_tests}
\end{table}

\newpage
\subsubsection{Student Database Tests}

\begin{table}[h]
    \centering
    \renewcommand{\arraystretch}{1.3}
    \begin{tabular}{|p{5cm}|p{5cm}|p{4cm}|c|}
        \hline
        \textbf{Test Case} & \textbf{Description} & \textbf{Expected Outcome} & \textbf{Status} \\
        \hline
        Insert student & Add a new student record & \ding{51} Successfully inserted & \ding{51} Pass \\
        \hline
        Retrieve by ID & Fetch student details & \ding{51} Retrieves correct data & \ding{51} Pass \\
        \hline
        Retrieve by email & Fetch student using email & \ding{51} Retrieves correct data & \ding{51} Pass \\
        \hline
        Update student profile & Modify student information & \ding{51} Updates successfully & \ding{51} Pass \\
        \hline
    \end{tabular}
    \caption{Student Database Test Cases}
    \label{tab:student_database_tests}
\end{table}

\newpage
\subsubsection{University Database Tests}

\begin{table}[h]
    \centering
    \renewcommand{\arraystretch}{1.3}
    \begin{tabular}{|p{5cm}|p{5cm}|p{4cm}|c|}
        \hline
        \textbf{Test Case} & \textbf{Description} & \textbf{Expected Outcome} & \textbf{Status} \\
        \hline
        Insert university & Add a new university record & \ding{51} Successfully inserted & \ding{51} Pass \\
        \hline
        Retrieve by ID & Fetch university details & \ding{51} Retrieves correct data & \ding{51} Pass \\
        \hline
        Retrieve by email & Fetch university using email & \ding{51} Retrieves correct data & \ding{51} Pass \\
        \hline
        Update university details & Modify university profile information & \ding{51} Updates successfully & \ding{51} Pass \\
        \hline
    \end{tabular}
    \caption{University Database Test Cases}
    \label{tab:university_database_tests}
\end{table}

\newpage
\subsubsection{User Database Tests}

\begin{table}[h]
    \centering
    \renewcommand{\arraystretch}{1.3}
    \begin{tabular}{|p{5cm}|p{5cm}|p{4cm}|c|}
        \hline
        \textbf{Test Case} & \textbf{Description} & \textbf{Expected Outcome} & \textbf{Status} \\
        \hline
        Insert user & Add a new user record & \ding{51} Successfully inserted & \ding{51} Pass \\
        \hline
        Duplicate user insert & Attempt to insert duplicate email & \ding{55} Raises exception & \ding{51} Pass \\
        \hline
        Check email uniqueness & Verify if an email is unique before registration & \ding{51} Returns correct boolean & \ding{51} Pass \\
        \hline
        Retrieve user type by email & Fetch account type of a user & \ding{51} Retrieves correct type & \ding{51} Pass \\
        \hline
        Retrieve user type by ID & Fetch account type using user ID & \ding{51} Retrieves correct type & \ding{51} Pass \\
        \hline
    \end{tabular}
    \caption{User Database Test Cases}
    \label{tab:user_database_tests}
\end{table}

\newpage
\subsection{Model Tests}

The following tests verify the model functionality within the \textbf{Students \& Companies (S\&C) platform}. These tests check class attributes, getters, setters, conversions to dictionary format, and integration with the database layer.

\subsubsection{Assessment Model Tests}

\begin{table}[h]
    \centering
    \renewcommand{\arraystretch}{1.3}
    \begin{tabular}{|p{5cm}|p{5cm}|p{4cm}|c|}
        \hline
        \textbf{Test Case} & \textbf{Description} & \textbf{Expected Outcome} & \textbf{Status} \\
        \hline
        Get assessment ID & Retrieve the assessment ID & \ding{51} Returns correct ID & \ding{51} Pass \\
        \hline
        Get application ID & Retrieve associated application ID & \ding{51} Returns correct application ID & \ding{51} Pass \\
        \hline
        Get date & Retrieve assessment date & \ding{51} Returns correct date & \ding{51} Pass \\
        \hline
        Get link & Retrieve assessment link & \ding{51} Returns correct link & \ding{51} Pass \\
        \hline
        Convert to dictionary & Convert assessment object to dictionary format & \ding{51} Returns correct dict & \ding{51} Pass \\
        \hline
        Add assessment (valid) & Insert a new assessment & \ding{51} Successfully inserted & \ding{51} Pass \\
        \hline
        Add assessment (invalid) & Insert an assessment with error & \ding{55} Returns None & \ding{51} Pass \\
        \hline
        Get last assessment by application ID (valid) & Retrieve last assessment for a valid application & \ding{51} Returns correct assessment & \ding{51} Pass \\
        \hline
        Get last assessment by application ID (invalid) & Retrieve last assessment for a non-existent application & \ding{55} Returns None & \ding{51} Pass \\
        \hline
        Handle exception in get last assessment & Simulate an exception scenario & \ding{55} Raises exception & \ding{51} Pass \\
        \hline
    \end{tabular}
    \caption{Assessment Model Test Cases}
    \label{tab:assessment_model_tests}
\end{table}

\newpage
\subsubsection{Application Model Tests}

\begin{table}[h]
    \centering
    \renewcommand{\arraystretch}{1.3}
    \begin{tabular}{|p{5cm}|p{5cm}|p{4cm}|c|}
        \hline
        \textbf{Test Case} & \textbf{Description} & \textbf{Expected Outcome} & \textbf{Status} \\
        \hline
        Get application ID & Retrieve application ID & \ding{51} Returns correct ID & \ding{51} Pass \\
        \hline
        Get student ID & Retrieve associated student ID & \ding{51} Returns correct ID & \ding{51} Pass \\
        \hline
        Get internship position ID & Retrieve associated internship position ID & \ding{51} Returns correct ID & \ding{51} Pass \\
        \hline
        Get status & Retrieve application status & \ding{51} Returns correct status & \ding{51} Pass \\
        \hline
        Convert to dictionary & Convert application object to dictionary format & \ding{51} Returns correct dict & \ding{51} Pass \\
        \hline
        Add application (valid) & Insert a new application & \ding{51} Successfully inserted & \ding{51} Pass \\
        \hline
        Add application (invalid) & Insert application with missing fields & \ding{55} Returns None & \ding{51} Pass \\
        \hline
        Retrieve by ID (valid) & Fetch application by valid ID & \ding{51} Returns correct data & \ding{51} Pass \\
        \hline
        Retrieve by ID (invalid) & Fetch application by non-existent ID & \ding{55} Returns None & \ding{51} Pass \\
        \hline
        Handle exception in get by ID & Simulate an exception in retrieval & \ding{55} Raises exception & \ding{51} Pass \\
        \hline
    \end{tabular}
    \caption{Application Model Test Cases}
    \label{tab:application_model_tests}
\end{table}

\newpage
\subsubsection{University Model Tests}

\begin{table}[h]
    \centering
    \renewcommand{\arraystretch}{1.3}
    \begin{tabular}{|p{5cm}|p{5cm}|p{4cm}|c|}
        \hline
        \textbf{Test Case} & \textbf{Description} & \textbf{Expected Outcome} & \textbf{Status} \\
        \hline
        Get university ID & Retrieve the university ID & \ding{51} Returns correct ID & \ding{51} Pass \\
        \hline
        Get email & Retrieve email of the university & \ding{51} Returns correct email & \ding{51} Pass \\
        \hline
        Get university name & Retrieve the name of the university & \ding{51} Returns correct name & \ding{51} Pass \\
        \hline
        Convert to dictionary & Convert university object to dictionary format & \ding{51} Returns correct dict & \ding{51} Pass \\
        \hline
        Add university (valid) & Insert a new university & \ding{51} Successfully inserted & \ding{51} Pass \\
        \hline
        Add university (invalid) & Insert a university with missing fields & \ding{55} Returns None & \ding{51} Pass \\
        \hline
    \end{tabular}
    \caption{University Model Test Cases}
    \label{tab:university_model_tests}
\end{table}

\newpage
\subsubsection{User Model Tests}

\begin{table}[h]
    \centering
    \renewcommand{\arraystretch}{1.3}
    \begin{tabular}{|p{5cm}|p{5cm}|p{4cm}|c|}
        \hline
        \textbf{Test Case} & \textbf{Description} & \textbf{Expected Outcome} & \textbf{Status} \\
        \hline
        Get user ID & Retrieve user ID & \ding{51} Returns correct ID & \ding{51} Pass \\
        \hline
        Get email & Retrieve user email & \ding{51} Returns correct email & \ding{51} Pass \\
        \hline
        Get user type & Retrieve user type (e.g., student, company) & \ding{51} Returns correct type & \ding{51} Pass \\
        \hline
        Validate correct password & Check correct password validation & \ding{51} Returns True & \ding{51} Pass \\
        \hline
        Validate incorrect password & Check incorrect password validation & \ding{55} Returns False & \ding{51} Pass \\
        \hline
        Get user type by email (valid) & Retrieve user type using email & \ding{51} Returns correct type & \ding{51} Pass \\
        \hline
    \end{tabular}
    \caption{User Model Test Cases}
    \label{tab:user_model_tests}
\end{table}

\subsection{Authentication Service Tests}

These tests validate authentication-related functionalities, including validation of different input types, email uniqueness, and authentication attributes.

\begin{longtable}{|p{5cm}|p{6cm}|p{3.5cm}|c|}
    \hline
    \textbf{Test Case} & \textbf{Description} & \textbf{Expected Outcome} & \textbf{Status} \\
    \hline
    \endfirsthead
    
    \hline
    \textbf{Test Case} & \textbf{Description} & \textbf{Expected Outcome} & \textbf{Status} \\
    \hline
    \endhead
    
    \hline
    \multicolumn{4}{|r|}{{Continued on next page}} \\
    \hline
    \endfoot
    
    \hline
    \endlastfoot

    String validation (valid) & Check if a valid string is detected correctly & \ding{51} Returns True & \ding{51} Pass \\
    \hline
    String validation (invalid) & Check if an integer is incorrectly treated as a string & \ding{55} Returns False & \ding{51} Pass \\
    \hline
    Integer validation (valid) & Validate correct integer inputs & \ding{51} Returns True & \ding{51} Pass \\
    \hline
    Integer validation (invalid) & Validate incorrect non-integer inputs & \ding{55} Returns False & \ding{51} Pass \\
    \hline
    Float validation (valid) & Validate correct float inputs & \ding{51} Returns True & \ding{51} Pass \\
    \hline
    Float validation (invalid) & Validate incorrect non-float inputs & \ding{55} Returns False & \ding{51} Pass \\
    \hline
    Email uniqueness check (valid) & Check if an unused email is detected correctly & \ding{51} Returns True & \ding{51} Pass \\
    \hline
    Email uniqueness check (invalid) & Check if an existing email is detected correctly & \ding{55} Returns False & \ding{51} Pass \\
    \hline
    Email uniqueness check (exception) & Handle exceptions during uniqueness check & \ding{55} Raises exception & \ding{51} Pass \\
    \hline
    Email format validation (valid) & Validate correct email formats & \ding{51} Returns True & \ding{51} Pass \\
    \hline
    Email format validation (invalid) & Validate incorrect email formats & \ding{55} Returns False & \ding{51} Pass \\
    \hline
    Phone number validation (valid) & Validate correct phone number formats & \ding{51} Returns True & \ding{51} Pass \\
    \hline
    Phone number validation (invalid) & Validate incorrect phone number formats & \ding{55} Returns False & \ding{51} Pass \\
    \hline
    Password validation (valid) & Validate strong passwords & \ding{51} Returns True & \ding{51} Pass \\
    \hline
    Password validation (invalid) & Validate weak passwords & \ding{55} Returns False & \ding{51} Pass \\
    \hline
    URL validation (valid) & Validate correct URLs & \ding{51} Returns True & \ding{51} Pass \\
    \hline
    URL validation (invalid) & Validate incorrect URLs & \ding{55} Returns False & \ding{51} Pass \\
    \hline
    Location validation (valid) & Validate correct locations & \ding{51} Returns True & \ding{51} Pass \\
    \hline
    Location validation (invalid) & Validate incorrect locations & \ding{55} Returns False & \ding{51} Pass \\
    \hline
    Name validation (valid) & Validate correct names & \ding{51} Returns True & \ding{51} Pass \\
    \hline
    Name validation (invalid) & Validate incorrect names & \ding{55} Returns False & \ding{51} Pass \\
    \hline
    Degree program validation (valid) & Validate correct degree programs & \ding{51} Returns True & \ding{51} Pass \\
    \hline
    Degree program validation (invalid) & Validate incorrect degree programs & \ding{55} Returns False & \ding{51} Pass \\
    \hline
    GPA validation (valid) & Validate correct GPA values & \ding{51} Returns True & \ding{51} Pass \\
    \hline
    GPA validation (invalid) & Validate incorrect GPA values & \ding{55} Returns False & \ding{51} Pass \\
    \hline
    Graduation year validation (valid) & Validate correct graduation years & \ding{51} Returns True & \ding{51} Pass \\
    \hline
    Graduation year validation (invalid) & Validate incorrect graduation years & \ding{55} Returns False & \ding{51} Pass \\
    \hline
    Path validation (valid) & Validate correct paths & \ding{51} Returns True & \ding{51} Pass \\
    \hline
    Path validation (invalid) & Validate incorrect paths & \ding{55} Returns False & \ding{51} Pass \\
    \hline
    Optional path validation (valid) & Validate correct optional paths & \ding{51} Returns True & \ding{51} Pass \\
    \hline
    Optional path validation (invalid) & Validate incorrect optional paths & \ding{55} Returns False & \ding{51} Pass \\
    \hline
\end{longtable}
\caption{Authentication Service Test Cases}
\label{tab:authentication_service_tests}

\section{Additional Notes}

The code is well-structured, with thorough validation tests covering various input scenarios. It ensures proper handling of valid and invalid data types across different authentication and validation functions.

The test suite effectively verifies different edge cases, exceptions, and expected behaviors, demonstrating a responsible and systematic approach to software quality assurance.

The immediate solution to any encountered issues, such as downloading individual data, highlights the team's problem-solving skills and commitment to collaboration.

The tested group was highly responsive, addressing queries with prompt and detailed explanations, showing a strong understanding of their system.

\section{Areas for Improvement}

\subsection{Meta-Testing (Tests for Tests)}
While the current test suite is comprehensive, additional meta-tests could be implemented to verify the effectiveness and completeness of the test cases themselves. This would help ensure that:
\begin{itemize}
    \item Test functions actually cover all possible edge cases.
    \item Expected failures occur where intended.
    \item Mocks and patches behave correctly.
\end{itemize}

Implementing **meta-testing** would add an extra layer of validation to the test framework, reinforcing the reliability and robustness of the software testing process.


     \chapter{Effort Spent}
    \label{ch:effort_spents}%
    \begin{table}[H]
    \centering
    \begin{tabular}{|p{0.25\textwidth}|p{0.5\textwidth}|c|}
        \hline
        \textbf{Team Member} & \textbf{Task} & \textbf{Hours Spent} \\ 
        \hline
        Shreesh Kumar Jha & 
        \begin{enumerate}
            \item Introduction and Document Structure
            \item Installation Setup and Backend Installation
            \item Test Case Mapping to Requirements
            \item Additional Notes and Areas for Improvement
        \end{enumerate} & 12 \\ 
        \hline
        Samarth Bhatia & 
        \begin{enumerate}
            \item Installation Setup and Frontend Installation
            \item Testing Environment and API Testing
            \item Backend and Database Testing
            \item Authentication Service and Validation Tests
        \end{enumerate} & 11 \\ 
        \hline
        Satvik Sharma & 
        \begin{enumerate}
            \item Model Testing and Meta-Testing
            \item Effort Spent Calculation and Formatting
        \end{enumerate} & 6 \\ 
        \hline
    \end{tabular}
    \caption{Effort spent by each member of the group.}
    \label{tab:effort_spent}
\end{table}


    \chapter{References}
    \label{ch:references}%
    \section{References}
\label{sec:references}%

\begin{itemize}
    \item Software Engineering 2 Course Materials, A.Y. 2024-2025.
    \item Daniel Jackson, \textit{Software Abstractions: Logic, Language, and Analysis}.
    \item Assignment RDD AY 2024-2025.pdf.
    \item  MongoDB, Inc. (n.d.). \href{https://www.mongodb.com/docs/manual/}{MongoDB Manual}.
    \item Node.js Foundation. (n.d.). \href{https://nodejs.org/en/docs/}{Node.js Documentation.}

\end{itemize}


% LIST OF TABLES
    \listoftables
    \cleardoublepage


\end{document}
