\documentclass{ConfigurationFiles/Politecnico_Di_Milano}


% CONFIGURATIONS
\usepackage{parskip} % For 
\usepackage{booktabs}
\usepackage{xcolor}
%paragraph layout
\usepackage{setspace} % For using single or double spacing
\usepackage{emptypage} % To insert empty pages
\usepackage{multicol} % To write in multiple columns (executive summary)
\setlength\columnsep{15pt} % Column separation in executive summary
\setlength\parindent{0pt} % Indentation
\raggedbottom
\usepackage{multirow}

% PACKAGES FOR TITLES
\usepackage{titlesec}
% \titlespacing{\section}{left spacing}{before spacing}{after spacing}
\titlespacing{\section}{0pt}{3.3ex}{2ex}
\titlespacing{\subsection}{0pt}{3.3ex}{1.65ex}
\titlespacing{\subsubsection}{0pt}{3.3ex}{1ex}
\usepackage{color}
\usepackage{enumitem}


\usepackage{longtable}            % For multipage tables
\usepackage{array}                % For p{width} column types
\usepackage{booktabs}             % (Optional) for nicer rules
\usepackage{lmodern}              %
\usepackage{pdflscape}
% PACKAGES FOR LANGUAGE AND FONT
\usepackage[english]{babel} % The document is in English  
\usepackage[utf8]{inputenc} % UTF8 encoding
\usepackage[T1]{fontenc} % Font encoding
\usepackage[11pt]{moresize} % Big fonts

% PACKAGES FOR IMAGES
\usepackage{graphicx}
\usepackage{transparent} % Enables transparent images
\usepackage{eso-pic} % For the background picture on the title page
\usepackage{subfig} % Numbered and caption subfigures using \subfloat.
\usepackage{tikz} % A package for high-quality hand-made figures.
\usetikzlibrary{}
\graphicspath{{./images/}} % Directory of the images
\usepackage{amsthm} % Coloured "Theorem"
\usepackage{thmtools}
\usepackage{xcolor}
\usepackage{float}

% STANDARD MATH PACKAGES
\usepackage{amsmath}
\usepackage{amssymb}
\usepackage{amsfonts}
\usepackage{bm}
\usepackage[overload]{empheq} % For braced-style systems of equations.
\usepackage{fix-cm} % To override original LaTeX restrictions on sizes

% PACKAGES FOR TABLES
\usepackage{tabularx}
\usepackage{longtable} % Tables that can span several pages
\usepackage{colortbl}

% PACKAGES FOR ALGORITHMS (PSEUDO-CODE)
\usepackage{algorithm}
\usepackage{algorithmic}

\usepackage{pifont}

% PACKAGES FOR REFERENCES & BIBLIOGRAPHY
\usepackage[
    colorlinks=true,
    linkcolor=black,
    anchorcolor=black,
    citecolor=black,
    filecolor=black,
    menucolor=black,
    runcolor=black,
    urlcolor=black
]{hyperref} % Adds clickable links at references
\usepackage{cleveref}
\usepackage[square, numbers, sort&compress]{natbib} % Square brackets, citing references with numbers, citations sorted by appearance in the text and compressed
\bibliographystyle{abbrvnat} % You may use a different style adapted to your field

% OTHER PACKAGES
\usepackage{pdfpages} % To include a pdf file
\usepackage{afterpage}
\usepackage{lipsum} % DUMMY PACKAGE
\usepackage{fancyhdr}
\usepackage{wasysym} % For the headers
\usepackage{rotating}
\usepackage{listings}
\usepackage{hyperref}
%%
% Alloy language definition for using with the listings package.
%
% 2017, Daniel Andrade
% BSD 3-Clause License
%%
\lstdefinelanguage{alloy}{
    morekeywords={
        module, open, as,
        private, abstract, sig, extends, in,
        lone, some, one, disj,
        fact, pred, fun, assert,
        run, check,
        for, but, exactly,
        this, not, implies, else, let,
        not, no, set, all, sum,
        iff, or, Int, and,
        none, univ, iden
    },
    sensitive=true,
    morecomment=[l]{//},
    morecomment=[l]{--},
    morecomment=[s]{/*}{*/},
    morestring=[b]{"},
%literate={->}{$\rightarrow$}1
% replacing characters can cause problems when copying from PDF to editor
}[keywords,comments,strings]

\fancyhf{}

% Input of configuration file. Do not change config.tex file unless you really know what you are doing. 
% Define dark magenta color
\definecolor{darkmagenta}{rgb}{0.6, 0.0, 0.6}

% Custom theorem environments
\declaretheoremstyle[
    headfont=\color{darkmagenta}\normalfont\bfseries,
    bodyfont=\color{black}\normalfont\itshape,
]{colored}

% Set-up caption colors
\captionsetup[figure]{labelfont={color=darkmagenta}} % Set colour of the captions
\captionsetup[table]{labelfont={color=darkmagenta}} % Set colour of the captions
\captionsetup[algorithm]{labelfont={color=darkmagenta}} % Set colour of the captions

\theoremstyle{colored}
\newtheorem{theorem}{Theorem}[chapter]
\newtheorem{proposition}{Proposition}[chapter]

% Enhances the features of the standard "table" and "tabular" environments.
\newcommand\T{\rule{0pt}{2.6ex}}
\newcommand\B{\rule[-1.2ex]{0pt}{0pt}}

% Pseudo-code algorithm descriptions.
\newcounter{algsubstate}
\renewcommand{\thealgsubstate}{\alph{algsubstate}}
\newenvironment{algsubstates}
{\setcounter{algsubstate}{0}%
\renewcommand{\STATE}{%
    \stepcounter{algsubstate}%
    \Statex {\small\thealgsubstate:}\space}}
{}

% New font size
\newcommand\numfontsize{\@setfontsize\Huge{200}{60}}

% Title format: chapter
\titleformat{\chapter}[hang]{
    \fontsize{50}{20}\selectfont\bfseries\filright}{\textcolor{darkmagenta} \thechapter\hsp\hspace{2mm}\textcolor{darkmagenta}{|   }\hsp}{0pt}{\huge\bfseries \textcolor{darkmagenta}
}

% Title format: section
\titleformat{\section}
{\color{darkmagenta}\normalfont\Large\bfseries}
{\color{darkmagenta}\thesection.}{1em}{}

% Title format: subsection
\titleformat{\subsection}
{\color{darkmagenta}\normalfont\large\bfseries}
{\color{darkmagenta}\thesubsection.}{1em}{}

% Title format: subsubsection
\titleformat{\subsubsection}
{\color{darkmagenta}\normalfont\large\bfseries}
{\color{darkmagenta}\thesubsubsection.}{1em}{}

% Shortening for setting no horizontal-spacing
\newcommand{\hsp}{\hspace{0pt}}

\makeatletter
% Renewcommand: cleardoublepage including the background pic
\renewcommand*\cleardoublepage{%
    \clearpage\if@twoside\ifodd\c@page\else
    \null
    \AddToShipoutPicture*{\BackgroundPic}
    \thispagestyle{empty}%
    \newpage
    \if@twocolumn\hbox{}\newpage\fi\fi\fi}
\makeatother

%For correctly numbering algorithms
\numberwithin{algorithm}{chapter}



\definecolor{dkgreen}{rgb}{0,0.6,0}
\definecolor{gray}{rgb}{0.5,0.5,0.5}
\definecolor{mauve}{rgb}{0.58,0,0.82}

\lstset{frame=tb,
    language=alloy,
    aboveskip=3mm,
    belowskip=3mm,
    showstringspaces=false,
    columns=flexible,
    basicstyle={\small\ttfamily},
    numbers=none,
    numberstyle=\tiny\color{gray},
    keywordstyle=\bf\color{blue},
    commentstyle=\it\color{dkgreen},
    stringstyle=\color{mauve},
    breaklines=true,
    breakatwhitespace=true,
    tabsize=3
}







%----------------------------------------------------------------------------
%	BEGIN OF YOUR DOCUMENT
%----------------------------------------------------------------------------



\begin{document}
    \fancypagestyle{plain}{%
        \fancyhf{} % Clear all header and footer fields
        \fancyhead[RO,RE]{\thepage} %RO=right odd, RE=right even
        \renewcommand{\headrulewidth}{0pt}
        \renewcommand{\footrulewidth}{0pt}}

        
    \pagestyle{empty} % No page numbers
    \frontmatter % Use roman page numbering style (i, ii, iii, iv...) for the preamble pages

    \puttitle{
        title=Software Engineering 2\\Acceptance Testing Documentation,
        name1=Shreesh Kumar Jha - 11022306, % Author Name and Surname
        name2=Samarth Bhatia - 11059097,
        name3=Satvik Sharma - 11054680,
        academicyear=2024-2025,
        version=2.0,
        releasedate=6/02/2025,
    }
    
    
    \startpreamble
    \setcounter{page}{1} % Set page counter to 1


% TABLE OF CONTENTS
    \thispagestyle{empty}
    \tableofcontents % Table of contents
    \thispagestyle{empty}
    \cleardoublepage

    
    \addtocontents{toc}{\vspace{1em}} % Add a gap in the Contents, for aesthetics
    \mainmatter % Begin numeric (1,2,3...) page numbering


    \chapter{Introduction}
    \label{ch:introduction}%
    \section{Purpose}
\label{sec:purpose}%
The Students \& Companies (S\&C) platform was created by a different team as a component of the Software Engineering 2 course, and the Acceptance Test Document (ATD) attempts to verify and methodically test its implementation. This document guarantees that the project satisfies the requirements listed in the Design Document (DD) and the Requirement Analysis and Specification Document (RASD). Verifying essential features, identifying discrepancies, and assessing how well the system satisfies user requirements and expected behaviors are the main goals of the testing process.


\section{Scope}
\label{subsec:scope}%
\newcounter{g}
\setcounter{g}{1}
\newcommand{\cg}{\theg\stepcounter{g}}

The Students \& Companies (S\&C) platform is designed to facilitate internship opportunities for students by connecting them with potential recruiters and enabling universities to monitor student internships. The platform includes the following key functionalities:
\begin{itemize}
    \item \textbf{Internship Lookup for Students:} Depending on their preferences, education, and skill set, students might look for and apply for internships.
    \item \textbf{Visibility for Companies:} Recruiters can post internship openings and match qualified student profiles with them.
    \item \textbf{Selection Process Management:} Employers are able to manage selections, set up interviews, and evaluate student applications.
    \item \textbf{Recommendation System:} Makes individualized internship recommendations using user data and gathered statistics.
     \item \textbf{Communication \& Feedback:} Allows for message sharing, problem reporting, and post-internship feedback between students and businesses.
     \item \textbf{University Monitoring:} To guarantee academic compliance, universities might supervise their students' internships.
\end{itemize}

The S\&C platform was developed by:
\begin{itemize}
    \item Edoardo S. Gribaldo
    \item Federico Rosa
\end{itemize}

The GitHub repository for the project can be found here: \href{https://github.com/edogriba/GribaldoRosa}{S\&C Repository}

\section{Revision History}
\label{sec:revision_history}

\begin{table}[H]
    \centering
    \begin{tabular}{ |p{0.1\linewidth}|p{0.15\linewidth}|p{0.45\linewidth}|p{0.2\linewidth}| }
        \hline
        \textbf{Version} & \textbf{Date} & \textbf{Description} & \textbf{Authors} \\
        \hline
        1.0 & 06 February 2025 & Initial Release & 
        Shreesh Kumar Jha, \newline
        Samarth Bhatia \\
        \hline
    \end{tabular}
    \caption{Revision History}
    \label{tab:revision_history}
\end{table}

\section{Reference Documents}
\label{sec:reference_documents}%
\begin{itemize}
    \item Reference to Previous Year Student Projects for Structuring the Document
    \item \href{https://github.com/edogriba/GribaldoRosa/blob/main/DeliveryFolder/RASDv1.pdf}{\textbf{\textcolor{blue}{\underline{RASDv1}}}}
    \item \href{https://github.com/edogriba/GribaldoRosa/blob/main/DeliveryFolder/ITDv1.pdf}{\textbf{\textcolor{blue}{\underline{ITDv1}}}}
    \item \href{https://github.com/edogriba/GribaldoRosa/blob/main/DeliveryFolder/DDv1.pdf}{\textbf{\textcolor{blue}{\underline{DDv1}}}}
\end{itemize}

\section{Document Structure}
\label{sec:document_structure}%
This document is structured as follows:

\subsection*{Installation Setup}
\begin{itemize}
    \item Details the steps followed to install and test the S\&C platform.
    \item Highlights any encountered issues, inconsistencies, or missing documentation.
\end{itemize}

\subsection*{Acceptance Test Cases}
\begin{itemize}
    \item Describes the test cases executed against the system.
    \item Maps tests to system requirements and goals as outlined in the \texttt{RASDv1}, \texttt{DDv1}, and \textttt{ITDv1}.
    \item Documents observed results, including any failures and potential fixes.
\end{itemize}


    \chapter{Implemented Requirements}
    \label{ch:architectural_design}%
    \section{Overview}
\label{sec:overview}
This section describes the functionalities that have been implemented in relation to the requirements specified in the RASD paper. Every feature of the system is mentioned below, along with an explanation of how to handle the database, front end, and back end.

\section{User Management}
\subsection*{Implemented Requirements}
\begin{itemize}
    \item \textbf{[F1.1]} : The system allows students, companies, and university administrators to register with verified email addresses.
    \item \textbf{[F1.3]} : The system allows users to update their profile information, including contact details and preferences.
    \item \textbf{[F1.4]} : The system enforces role-based access control for students, companies, and administrators.
    \item \textbf{[F1.5]} : The system supports password reset functionality with email verification.
    \item \textbf{[F1.6]} : The system maintains audit logs of all user authentication activities.
    \item \textbf{[F1.7]} : The system allows users to manage notification preferences.
    \item \textbf{[F1.8]} : The system enforces strong password policies.
\end{itemize}

\subsection*{Non-Implemented Requirements}
\begin{itemize}
    \item \textbf{[F1.2]} : Secure authentication with optional two-factor verification has not been implemented due to time constraints and the additional complexity of integrating multi-factor authentication systems at this stage of development.
\end{itemize}

\subsection*{Database}
\paragraph{Tables:}
\begin{itemize}
    \item \texttt{users}: Stores user details, including email, password, role, and status.
    \item \texttt{user\_profiles}\textbf{(Student, Recruiter, Admin)}: Contains user-specific details such as contact information and preferences.
    \item \texttt{notification\_preferences}: Stores user preferences for managing notifications.
\end{itemize}

\paragraph{Constraints:}
\begin{itemize}
    \item Emails are unique across all users and validated during registration.
    \item Passwords are hashed and salted using bcrypt before storage.
    \item Role-based constraints ensure that data integrity is maintained for each user type (student, company, administrator).
\end{itemize}

\subsection*{Back-end}
\paragraph{Controllers:}
\begin{itemize}
    \item \texttt{userController.js}:
    \begin{itemize}
        \item Handles user registration (\texttt{POST /register}), login (\texttt{POST /login}), and password reset functionality (\texttt{POST /reset-password}).
        \item Verifies email addresses during registration by sending an email with a verification token.
    \end{itemize}
    \item \texttt{studentController.js}, \texttt{recruiterController.js}, \texttt{adminController.js}:
    \begin{itemize}
        \item Role-specific logic for profile updates and account management.
    \end{itemize}
\end{itemize}

\paragraph{Functions:}
\begin{itemize}
    \item \textbf{Registration Validation:} Ensures that all required fields (e.g., email, password, role) are provided and correctly formatted.
    \item \textbf{Role Enforcement:} Restricts access to certain endpoints based on the role field stored in JWT tokens.
\end{itemize}

\subsection*{Front-end}
\paragraph{Application and Website:}
\begin{itemize}
    \item \textbf{Registration:} Students, companies, and administrators register using forms with dynamic validation feedback.
    \item \textbf{Profile Updates:} Users can update personal details (e.g., contact information) through a dedicated profile section.
    \item \textbf{Password Reset:} Users initiate a password reset process, with password strength validation before submission.
    \item \textbf{Role-Based Dashboards:} Students, companies, and administrators are redirected to role-specific dashboards post-login.
\end{itemize}

\subsection*{Commentary on Non-Implemented Requirements}
\begin{itemize}
    \item \textbf{[F1.2]} : Multi-factor authentication was excluded due to the effort required for integrating and managing a third-party authentication service, such as Google Authenticator or Twilio for SMS. This feature is planned for future iterations to enhance security.
\end{itemize}

\section{CV Management}
\subsection*{Implemented Requirements}
\begin{itemize}
    \item \textbf{[F2.1]} : The system provides customizable CV templates for students.
    \item \textbf{[F2.2]} : Students can create and store multiple versions of their CVs.
    \item \textbf{[F2.3]} : Students can update their CVs at any time.
    \item \textbf{[F2.4]} : Students can control CV visibility to specific companies.
    \item \textbf{[F2.6]} : The system validates skill entries against a standardized skill database.
\end{itemize}

\subsection*{Non-Implemented Requirements}
\begin{itemize}
    \item [\textbf{F2.7]} : Document uploads (e.g., certificates, portfolios).
    \item \textbf{[F2.8]} : Tracking CV view statistics for students.
\end{itemize}

\subsection*{Database}
\paragraph{Model:} \texttt{Cv}
\paragraph{Fields:}
\begin{itemize}
    \item \textbf{user:} References the User model to associate the CV with a specific student.
    \item \textbf{template:} Stores the CV template identifier.
    \item \textbf{data:} Contains the detailed CV content, including sections like Education, Skills, and Experience.
    \item \textbf{visibility:} An array of company IDs defining which companies can view the CV.
\end{itemize}

\paragraph{Constraints:}
\begin{itemize}
    \item A student can have only one active CV at a time, with version updates stored dynamically.
    \item The \texttt{visibility} field ensures that only authorized companies can access the CV.
\end{itemize}

\subsection*{Back-End}
\paragraph{Routes:} Defined in \texttt{cvRoutes.js}
\begin{itemize}
    \item \texttt{POST /:} Creates or updates a CV for the logged-in user.
    \item \texttt{GET /latest:} Fetches the latest CV of the logged-in user.
    \item \texttt{DELETE /:} Deletes the logged-in user’s CV.
    \item \texttt{POST /update-visibility:} Updates the visibility settings for a CV.
    \item \texttt{GET /:studentId:} Retrieves a specific student’s CV by their ID.
\end{itemize}

\paragraph{Controllers:} Defined in \texttt{cvController.js}
\begin{itemize}
    \item \textbf{createOrUpdateCV:} Handles both creation and updates to a CV. If a CV already exists for a student, it updates the template, data, and visibility fields.
    \item \textbf{getCV:} Retrieves the logged-in user’s CV. If none exists, returns a 404 error.
    \item \textbf{deleteCV:} Deletes the logged-in user’s CV.
    \item \textbf{updateVisibility:} Updates the list of companies that can view a specific CV.
\end{itemize}

\subsection*{Front-End}
\paragraph{Application:}
\begin{itemize}
    \item \textbf{CV Builder:}
    \begin{itemize}
        \item Students can select templates and customize sections like Education, Skills, and Experience.
        \item Live previews of the CV are available during editing.
    \end{itemize}
    \item \textbf{Visibility Management:}
    \begin{itemize}
        \item A toggle interface allows students to manage company permissions for viewing their CVs.
    \end{itemize}
\end{itemize}

\paragraph{Website:}
\begin{itemize}
    \item The CV builder is integrated with the student dashboard.
    \item Visibility settings can be updated directly from the CV management panel.
\end{itemize}

\subsection*{Commentary on Non-Implemented Requirements}
\begin{itemize}
    \item \textbf{[F2.7]} Document upload functionality has not been implemented due to the complexity of file storage and retrieval. This feature is planned for future development.
    \item \textbf{[F2.8]} CV view statistics tracking is excluded as an analytics framework for monitoring company interactions has not yet been integrated.
\end{itemize}

\section{Internship Management}
\subsection*{Implemented Requirements}
\begin{itemize}
    \item \textbf{[F3.1]} : The system allows companies to create detailed internship postings.
    \item \textbf{[F3.2]} : Provides an application tracking system for companies.
    \item \textbf{[F3.3]} : Automatically notifies students of application status changes.
    \item \textbf{[F3.4]} : Provides advanced search and filtering capabilities.
    \item \textbf{[F3.6]} : Enables bulk application processing for companies.
    \item \textbf{[F3.7]} : Supports multiple rounds of application review.
    \item \textbf{[F3.8]} : Maintains a history of all internship postings.
\end{itemize}

\subsection*{Non-Implemented Requirements}
\begin{itemize}
    \item \textbf{[F3.5]} : Allowing companies to set application deadlines.
\end{itemize}

\subsection*{Database}
\paragraph{Model:} \texttt{Internship}
\paragraph{Fields:}
\begin{itemize}
    \item \textbf{Core Details:} \texttt{title}, \texttt{description}, \texttt{location}, \texttt{duration}, \texttt{stipend}.
    \item \textbf{Skills:} \texttt{requiredSkills} (mandatory), \texttt{preferredSkills} (optional).
    \item \textbf{Experience:} \texttt{experienceLevel} (e.g., beginner, intermediate, advanced).
    \item \textbf{Applicants:} Stores references to users who applied.
    \item \textbf{Interviews:} Includes details of scheduled interviews.
    \item \textbf{Status:} Tracks whether the internship is open, closed, or draft.
    \item \textbf{Timestamps:} \texttt{createdAt}, \texttt{updatedAt} fields auto-generated by Mongoose.
\end{itemize}

\subsection*{Back-End}
\paragraph{Routes:} Defined in \texttt{internshipRoutes.js}
\begin{itemize}
    \item \texttt{POST /addinternship:} Adds a new internship (role restricted to recruiters).
    \item \texttt{GET /allinternships:} Retrieves all internship postings.
    \item \texttt{POST /:id/apply:} Allows students to apply for internships.
    \item \texttt{GET /recommended:} Provides personalized internship recommendations for students.
    \item \texttt{PATCH /:internshipId/interviews/:studentId/completed:} Marks interviews as completed.
\end{itemize}

\paragraph{Controllers:}
\begin{itemize}
    \item \textbf{addInternship:} Handles creation of internships, validates inputs, and ensures required fields are present.
    \item \textbf{getAllInternships:} Fetches all internships, sorted by the latest postings.
    \item \textbf{applyToInternship:} Handles application submissions, ensuring no duplicate applications.
\end{itemize}

\subsection*{Front-End}
\paragraph{Application:}
\begin{itemize}
    \item \textbf{For Companies:} A guided form for creating internships with fields for \texttt{title}, \texttt{description}, \texttt{skills}, and experience requirements.
    \item \textbf{For Students:} Search and filter functionalities to find internships based on location, skills, and duration.
    \item \textbf{Dashboard Integration:} Displays a list of applied internships with their status.
\end{itemize}

\paragraph{Website:}
\begin{itemize}
    \item Applications dashboard for recruiters, showing real-time updates on applications received.
\end{itemize}

\subsection*{Commentary on Non-Implemented Requirements}
\begin{itemize}
    \item \textbf{[F3.5]} Application deadlines are not implemented in this version due to the additional logic required to enforce and validate deadlines during application submissions. This feature is planned for future iterations.
\end{itemize}

\section{Interview Management}
\subsection*{Implemented Requirements}
\begin{itemize}
    \item \textbf{[F4.1]} : The system provides a calendar interface for interview scheduling.
    \item \textbf{[F4.2]} : Tracks interview status and progress.
    \item \textbf{[F4.5]} : Supports virtual interview link generation.
    \item \textbf{[F4.6]} : Allows rescheduling with mutual agreement.
    \item \textbf{[F4.7]} : Maintains interview history.
    \item \textbf{[F4.8]} : Supports multiple interview rounds.
\end{itemize}

\subsection*{Non-Implemented Requirements}
\begin{itemize}
    \item \textbf{[F4.3]} : Structured feedback collection from both parties is not implemented.
    \item \textbf{[F4.4]} : Automated interview reminders are not supported.
\end{itemize}

\subsection*{Database}
\paragraph{Model:} \texttt{Internship}
\paragraph{Fields:}
\begin{itemize}
    \item \textbf{scheduledInterviews:} An array of interview objects, each containing:
    \begin{itemize}
        \item \textbf{student:} References the User model to identify the applicant.
        \item \textbf{dateTime:} Date and time of the interview.
        \item \textbf{meetLink:} Google Meet or other virtual link for the interview.
        \item \textbf{status:} Tracks the interview state (Scheduled or Completed).
    \end{itemize}
\end{itemize}

\paragraph{Model:} \texttt{Student}
\paragraph{Fields:}
\begin{itemize}
    \item \textbf{scheduledInterviews:} Tracks interviews scheduled for the student, referencing internship details and interview metadata.
\end{itemize}

\subsection*{Back-End}
\paragraph{Routes:} Defined in multiple controllers (\texttt{internshipRoutes.js}, \texttt{studentRoutes.js}, \texttt{recruiterRoutes.js}):
\begin{itemize}
    \item \texttt{POST /:id/schedule:} Schedules an interview for a student (role restricted to recruiters).
    \item \texttt{GET /student/interviews:} Retrieves scheduled interviews for students.
    \item \texttt{GET /recruiter/interviews:} Fetches interviews scheduled by recruiters.
    \item \texttt{PATCH /:internshipId/interviews/:studentId/completed:} Marks an interview as completed.
    \item \texttt{GET /student/completed-interviews:} Retrieves a student’s completed interviews.
\end{itemize}

\paragraph{Controllers:}
\begin{itemize}
    \item \textbf{scheduleInterview:} Generates a virtual link via Google Calendar API and saves interview details to both the internship and student profiles.
    \item \textbf{getRecruiterInterviews:} Fetches all interviews related to a recruiter’s internships.
    \item \textbf{getStudentInterviews:} Retrieves all scheduled interviews for a student.
    \item \textbf{markInterviewAsCompleted:} Updates the interview status to Completed.
\end{itemize}

\subsection*{Front-End}
\paragraph{Application:}
\begin{itemize}
    \item \textbf{Calendar Integration:}
    \begin{itemize}
        \item Students and recruiters can view interview schedules on an interactive calendar interface.
        \item Recruiters can reschedule interviews using drag-and-drop functionality.
    \end{itemize}
    \item \textbf{Interview History:}
    \begin{itemize}
        \item Students can view completed interviews with details such as company name, date, and virtual link.
    \end{itemize}
\end{itemize}

\paragraph{Website:}
\begin{itemize}
    \item Recruiters have a dashboard displaying all scheduled and completed interviews for their internships.
    \item Students can access a list of upcoming interviews along with relevant details.
\end{itemize}

\subsection*{Commentary on Non-Implemented Requirements}
\begin{itemize}
    \item \textbf{[F4.3]} : Feedback collection was deprioritized in this iteration due to the additional database design and UI components required for structured feedback forms.
    \item \textbf{[F4.4]} : Automated reminders for interviews were excluded to avoid the integration complexity of real-time notification systems. These will be considered in future versions.
\end{itemize}

\section{Recommendation System}
\subsection*{Implemented Requirements}
\begin{itemize}
    \item \textbf{[F5.1]} : Matches students with internships based on skills alignment.
    \item \textbf{[F5.2]} : Suggests qualified candidates to companies.
    \item \textbf{[F5.3]} : Learns from user interactions to improve recommendations.
    \item \textbf{[F5.4]} : Generates personalized internship suggestions.
    \item \textbf{[F5.5]} : Considers location preferences in matching.
    \item \textbf{[F5.6]} : Factors in past application success patterns.
    \item \textbf{[F5.7]} : Updates recommendations in real-time.
    \item \textbf{[F5.8]} : Explains recommendation reasoning to users.
\end{itemize}

\subsection*{Database}
\paragraph{Model:} \texttt{Internship}
\paragraph{Fields:}
\begin{itemize}
    \item \textbf{requiredSkills \& preferredSkills:} Define skill requirements for internships.
    \item \textbf{location:} Captures location data for matching against student preferences.
    \item \textbf{applicants:} Tracks users who have applied to the internship.
\end{itemize}

\paragraph{Model:} \texttt{Student}
\paragraph{Fields:}
\begin{itemize}
    \item \textbf{profile.skills:} Stores the student's skills for alignment with internship requirements.
    \item \textbf{profile.location:} Specifies location preferences.
    \item \textbf{appliedInternships:} Tracks internships the student has applied for.
\end{itemize}

\subsection*{Back-End}
\paragraph{Routes:} Defined in \texttt{internshipRoutes.js}
\begin{itemize}
    \item \texttt{GET /recommended:} Fetches recommended internships for a student based on their skills, location, and profile data.
\end{itemize}

\paragraph{Controllers:}
\begin{itemize}
    \item \textbf{getRecommendedInternships:}
    \begin{itemize}
        \item Fetches the student’s CV and profile data.
        \item Matches internships using weighted scoring:
        \begin{itemize}
            \item \textbf{Skills Alignment (50\%):} Compares student skills with required and preferred skills.
            \item \textbf{Experience Level (30\%):} Considers the total duration of relevant experiences.
            \item \textbf{Location Match (20\%):} Matches student’s preferred location with internship location.
        \end{itemize}
        \item Excludes internships already applied for by the student.
        \item Sorts and returns the top 10 recommendations with match scores and reasoning.
    \end{itemize}
    \item \textbf{calculateSkillMatch:} Compares student and internship skills, returning a similarity score and list of matching skills.
    \item \textbf{calculateLocationPreference:} Scores based on the student’s preferred and internship location match.
\end{itemize}

\subsection*{Front-End}
\paragraph{Application:}
\begin{itemize}
    \item \textbf{Dashboard Integration:}
    \begin{itemize}
        \item Displays personalized internship recommendations with match scores.
        \item Shows reasons for each recommendation, such as skill match or location preference.
    \end{itemize}
    \item \textbf{Real-Time Updates:} Recommendations dynamically update based on profile or application changes.
\end{itemize}

\paragraph{Website:}
\begin{itemize}
    \item Search results highlight recommended internships, ranked by match score.
    \item Filters allow students to refine recommendations based on additional criteria.
\end{itemize}

\section{Complaint Handling}
\subsection*{Implemented Requirements}
\begin{itemize}
    \item \textbf{[F6.1]} : Provides a structured complaint submission interface.
    \item \textbf{[F6.2]} : Routes complaints to appropriate university administrators.
    \item \textbf{[F6.3]} : Tracks resolution progress for each complaint.
    \item \textbf{[F6.4]} : Supports an appeals process.
    \item \textbf{[F6.5]} : Maintains a complete complaint history.
    \item \textbf{[F6.6]} : Enables communication between involved parties.
    \item \textbf{[F6.7]} : Generates complaint resolution reports.
    \item \textbf{[F6.8]} : Enforces resolution timeframes.
\end{itemize}

\subsection*{Database}
\paragraph{Model:} \texttt{Complaint}
\paragraph{Fields:}
\begin{itemize}
    \item \textbf{userId:} References the User model to identify the user who raised the complaint.
    \item \textbf{role:} Specifies the role of the user (student or recruiter).
    \item \textbf{title:} Title of the complaint for quick reference.
    \item \textbf{description:} Detailed description of the issue.
    \item \textbf{status:} Tracks the state of the complaint (pending or resolved).
    \item \textbf{createdAt:} Timestamp for when the complaint was created.
\end{itemize}

\subsection*{Back-End}
\paragraph{Routes:} Defined in \texttt{complaintRoutes.js}
\begin{itemize}
    \item \texttt{POST /create-complaint:} Allows students or recruiters to submit complaints.
    \item \texttt{GET /get-complaints:} Fetches all pending complaints (restricted to admins).
    \item \texttt{PATCH /:complaintId/resolve:} Marks a complaint as resolved (restricted to admins).
    \item \texttt{GET /my-complaints:} Retrieves complaints submitted by the logged-in user.
\end{itemize}

\paragraph{Controllers:} Defined in \texttt{complaintController.js}
\begin{itemize}
    \item \textbf{createComplaint:} Handles the creation of complaints, validates inputs, and stores them in the database.
    \item \textbf{getComplaints:} Retrieves all complaints with a pending status, populated with user details.
    \item \textbf{resolveComplaint:} Updates the status of a complaint to be resolved.
    \item \textbf{getMyComplaints:} Fetches complaints submitted by the logged-in user, allowing them to track their status.
\end{itemize}

\subsection*{Front-End}
\paragraph{Application:}
\begin{itemize}
    \item \textbf{Complaint Submission Interface:}
    \begin{itemize}
        \item Students and recruiters can submit complaints through a structured form with fields for the title and description.
    \end{itemize}
    \item \textbf{Complaint History:}
    \begin{itemize}
        \item Displays a list of all submitted complaints with their statuses.
        \item Provides real-time updates on resolution progress.
    \end{itemize}
\end{itemize}

\paragraph{Website:}
\begin{itemize}
    \item \textbf{Admin Dashboard:}
    \begin{itemize}
        \item Administrators can view, filter, and resolve complaints.
        \item Includes an interface for generating complaint resolution reports.
    \end{itemize}
\end{itemize}



    \chapter{Acceptance Test Case}
    \label{ch:user_interface_design}%
    \section{Acceptance Test Cases}

The \textbf{Students \& Companies (S\&C) platform} requirements are divided into two main categories:

\begin{itemize}
    \item \textbf{Core Requirements}: Fundamental system features that must be implemented and functional.
    \item \textbf{Goal Reaching Requirements}: Additional functionalities mapped to user goals, improving usability and efficiency.
\end{itemize}

The following sections outline some important test cases, linking each requirement (R1, R2, etc.) from the \textit{RASD} document to its corresponding test.  
\textbf{Overall: 23 Backend Tests.}

\newpage
\subsection{Core Requirements (Users)}

\begin{table}[h]
    \centering
    \renewcommand{\arraystretch}{1.3}
    \begin{tabular}{|c|p{4cm}|p{4cm}|p{4cm}|c|}
        \hline
        \textbf{Requirement} & \textbf{Test Description} & \textbf{Expected Outcome} & \textbf{Actual Outcome} & \textbf{Status} \\
        \hline
        R1 & Allow users to sign up. & Users can register successfully. & \ding{51} Works as expected. & \ding{51} Pass \\
        \hline
        R2 & Allow users to fill in profile information when signing up. & Profile data saved successfully. & \ding{51} Works as expected. & \ding{51} Pass \\
        \hline
        R3 & Allow users to log in. & Users can access their accounts. & \ding{51} Works as expected. & \ding{51} Pass \\
        \hline
        R4 & Allow users to log out. & Users are logged out securely. & \ding{51} Works as expected. & \ding{51} Pass \\
        \hline
        R5 & Allow users to update their profile information. & Profile updates are saved. & \ding{51} Works as expected. & \ding{51} Pass \\
        \hline
        R6 & Allow users to examine their own internships. & Users can see the internships they applied for. & \ding{51} Works as expected. & \ding{51} Pass \\
        \hline
    \end{tabular}
    \caption{Core Requirements Test Cases for Users}
    \label{tab:core_requirements}
\end{table}

\newpage
\subsection{Student Requirements}

\begin{table}[h]
    \centering
    \renewcommand{\arraystretch}{1.3}
    \begin{tabular}{|c|p{3cm}|p{3cm}|p{3cm}|c|}
        \hline
        \textbf{Requirement} & \textbf{Test Description} & \textbf{Expected Outcome} & \textbf{Actual Outcome} & \textbf{Status} \\
        \hline
        R7 & Allow students to examine open internship positions. & Available internships are listed. & \ding{51} Works as expected. & \ding{51} Pass \\
        \hline
        R8 & Allow students to examine their own applications. & Applications and statuses are visible. & \ding{51} Works as expected. & \ding{51} Pass \\
        \hline
        R9 & Allow students to search for a specific internship position. & Search results match the criteria. & \ding{51} Works as expected. & \ding{51} Pass \\
        \hline
        R10 & Allow students to apply for an internship position. & Application is submitted successfully. & \ding{51} Works as expected. & \ding{51} Pass \\
        \hline
        R11 & Notify students when a suitable internship is opened. & Students receive notifications for relevant internships. & \ding{51} Works as expected. & \ding{51} Pass \\
        \hline
        R12 & List different internship positions aligned with student profiles. & Internships displayed match student skills. & \ding{51} Works as expected. & \ding{51} Pass \\
        \hline
    \end{tabular}
    \caption{Student Requirements Test Cases}
    \label{tab:student_requirements}
\end{table}

\newpage
\subsection{Application Requirements}

\begin{table}[h]
    \centering
    \renewcommand{\arraystretch}{1.3}
    \begin{tabular}{|c|p{4cm}|p{4cm}|p{4.5cm}|c|}
        \hline
        \textbf{Requirement} & \textbf{Test Description} & \textbf{Expected Outcome} & \textbf{Actual Outcome} & \textbf{Status} \\
        \hline
        R13 & Allow students to confirm or refuse an accepted internship offer. & Students can accept/reject offers. & \ding{51} Works as expected. & \ding{51} Pass \\
        \hline
        R14 & Allow companies to request skill assessments and schedule interviews. & Recruiters can request interviews. & \ding{51} Works as expected. & \ding{51} Pass \\
        \hline
        R15 & Allow students to access interview details and links. & Interview details are available in the dashboard. & \ding{51} Works as expected. & \ding{51} Pass \\
        \hline
        R16 & Allow students to see the status of their applications. & Application status updates correctly. & \ding{51} Works as expected. & \ding{51} Pass \\
        \hline
    \end{tabular}
    \caption{Application Requirements Test Cases}
    \label{tab:application_requirements}
\end{table}

\newpage
\subsection{Company Requirements}

\begin{table}[h]
    \centering
    \renewcommand{\arraystretch}{1.3}
    \begin{tabular}{|c|p{4cm}|p{3.5cm}|p{4cm}|c|}
        \hline
        \textbf{Requirement} & \textbf{Test Description} & \textbf{Expected Outcome} & \textbf{Actual Outcome} & \textbf{Status} \\
        \hline
        R17 & Allow companies to open and examine internship positions. & Companies can create and view internships. & \ding{51} Works as expected. & \ding{51} Pass \\
        \hline
        R18 & Allow companies to accept or reject applications. & Recruiters can update application statuses. & \ding{51} Works as expected. & \ding{51} Pass \\
        \hline
        R19 & Allow companies to close internship positions. & Internship status updates to "Closed." & \ding{51} Works as expected. & \ding{51} Pass \\
        \hline
        R20 & Notify companies when a new relevant student profile is available. & Recruiters receive notifications about matching students. & \ding{51} Works as expected. & \ding{51} Pass \\
        \hline
    \end{tabular}
    \caption{Company Requirements Test Cases}
    \label{tab:company_requirements}
\end{table}

\newpage
\subsection{Feedback and Suggestions}
\begin{table}[h]
    \centering
    \renewcommand{\arraystretch}{1.3}
    \begin{tabular}{|c|p{4.5cm}|p{3.5cm}|p{4.5cm}|c|}
        \hline
        \textbf{Requirement} & \textbf{Test Description} & \textbf{Expected Outcome} & \textbf{Actual Outcome} & \textbf{Status} \\
        \hline
        R24 & Allow students and companies to rate the internship experience. & Ratings are recorded successfully. & \ding{51} Works as expected. & \ding{51} Pass \\
        \hline
        R25 & Allow students to provide feedback on the internship experience. & Students can submit feedback. & \ding{51} Works as expected. & \ding{51} Pass \\
        \hline
        R26 & Allow companies to send feedback/news to students. & Students receive feedback notifications. & \ding{51} Works as expected. & \ding{51} Pass \\
        \hline
        R27 & Allow both parties to file complaints about the internship. & Complaints are submitted and reviewed. & \ding{51} Works as expected. & \ding{51} Pass \\
        \hline
    \end{tabular}
    \caption{Feedback and Suggestions Test Cases}
    \label{tab:feedback_suggestions}
\end{table}

\subsection{Database Tests}

The following tests verify the database operations for different modules in the \textbf{Students \& Companies (S\&C) platform}. These tests check insertion, retrieval, updates, and constraints for key entities.

\subsubsection{Assessment Database Tests}

\begin{table}[h]
    \centering
    \renewcommand{\arraystretch}{1.3}
    \begin{tabular}{|p{5cm}|p{5cm}|p{4cm}|c|}
        \hline
        \textbf{Test Case} & \textbf{Description} & \textbf{Expected Outcome} & \textbf{Status} \\
        \hline
        Insert assessment & Add a new assessment record & \ding{51} Successfully inserted & \ding{51} Pass \\
        \hline
        Invalid insert & Insert invalid assessment data & \ding{55} Raises exception & \ding{51} Pass \\
        \hline
        Retrieve last assessment by application ID & Fetch last added assessment for an application & \ding{51} Retrieves latest assessment & \ding{51} Pass \\
        \hline
        Retrieve with invalid application ID & Query non-existent application ID & \ding{55} Returns None & \ding{51} Pass \\
        \hline
    \end{tabular}
    \caption{Assessment Database Test Cases}
    \label{tab:assessment_database_tests}
\end{table}

\subsubsection{Application Database Tests}

\begin{table}[h]
    \centering
    \renewcommand{\arraystretch}{1.3}
    \begin{tabular}{|p{5cm}|p{5cm}|p{4cm}|c|}
        \hline
        \textbf{Test Case} & \textbf{Description} & \textbf{Expected Outcome} & \textbf{Status} \\
        \hline
        Insert application & Add a new application entry & \ding{51} Successfully inserted & \ding{51} Pass \\
        \hline
        Invalid insert & Insert application with missing fields & \ding{55} Raises exception & \ding{51} Pass \\
        \hline
        Retrieve application by ID & Fetch an application record & \ding{51} Retrieves correct data & \ding{51} Pass \\
        \hline
        Retrieve application by student ID & Get applications for a student & \ding{51} Retrieves correct data & \ding{51} Pass \\
        \hline
        Retrieve application by internship ID & Get applications for an internship & \ding{51} Retrieves correct data & \ding{51} Pass \\
        \hline
        Update application status & Modify an application status & \ding{51} Status updates successfully & \ding{51} Pass \\
        \hline
    \end{tabular}
    \caption{Application Database Test Cases}
    \label{tab:application_database_tests}
\end{table}

\subsubsection{Company Database Tests}

\begin{table}[h]
    \centering
    \renewcommand{\arraystretch}{1.3}
    \begin{tabular}{|p{5cm}|p{5cm}|p{4cm}|c|}
        \hline
        \textbf{Test Case} & \textbf{Description} & \textbf{Expected Outcome} & \textbf{Status} \\
        \hline
        Insert company & Add a new company entry & \ding{51} Successfully inserted & \ding{51} Pass \\
        \hline
        Duplicate company insert & Try inserting a duplicate company & \ding{55} Raises exception & \ding{51} Pass \\
        \hline
        Retrieve company by ID & Fetch a company record & \ding{51} Retrieves correct data & \ding{51} Pass \\
        \hline
        Retrieve company by email & Fetch company details using email & \ding{51} Retrieves correct data & \ding{51} Pass \\
        \hline
        Update company details & Modify company profile data & \ding{51} Updates successfully & \ding{51} Pass \\
        \hline
    \end{tabular}
    \caption{Company Database Test Cases}
    \label{tab:company_database_tests}
\end{table}

\newpage
\subsubsection{Complaint Database Tests}

\begin{table}[h]
    \centering
    \renewcommand{\arraystretch}{1.3}
    \begin{tabular}{|p{5cm}|p{5cm}|p{4cm}|c|}
        \hline
        \textbf{Test Case} & \textbf{Description} & \textbf{Expected Outcome} & \textbf{Status} \\
        \hline
        Insert complaint & Add a new complaint entry & \ding{51} Successfully inserted & \ding{51} Pass \\
        \hline
        Invalid insert & Insert complaint with missing fields & \ding{55} Raises exception & \ding{51} Pass \\
        \hline
        Retrieve complaints by internship ID & Fetch complaints for an internship & \ding{51} Retrieves correct data & \ding{51} Pass \\
        \hline
        Retrieve complaints for non-existent internship & Query complaints for a non-existent ID & \ding{55} Returns empty list & \ding{51} Pass \\
        \hline
    \end{tabular}
    \caption{Complaint Database Test Cases}
    \label{tab:complaint_database_tests}
\end{table}

\subsubsection{Internship Database Tests}

\begin{table}[h]
    \centering
    \renewcommand{\arraystretch}{1.3}
    \begin{tabular}{|p{5cm}|p{5cm}|p{4cm}|c|}
        \hline
        \textbf{Test Case} & \textbf{Description} & \textbf{Expected Outcome} & \textbf{Status} \\
        \hline
        Insert internship & Add a new internship entry & \ding{51} Successfully inserted & \ding{51} Pass \\
        \hline
        Invalid insert & Insert internship with missing fields & \ding{55} Raises exception & \ding{51} Pass \\
        \hline
        Retrieve internship by ID & Fetch internship details & \ding{51} Retrieves correct data & \ding{51} Pass \\
        \hline
        Retrieve internship by application ID & Fetch internship for a given application & \ding{51} Retrieves correct data & \ding{51} Pass \\
        \hline
        Retrieve internships by company ID & Fetch all internships posted by a company & \ding{51} Retrieves correct data & \ding{51} Pass \\
        \hline
        Update internship status & Modify internship state (e.g., ongoing, finished) & \ding{51} Updates successfully & \ding{51} Pass \\
        \hline
    \end{tabular}
    \caption{Internship Database Test Cases}
    \label{tab:internship_database_tests}
\end{table}

\newpage
\subsubsection{Internship Position Database Tests}

\begin{table}[h]
    \centering
    \renewcommand{\arraystretch}{1.3}
    \begin{tabular}{|p{5cm}|p{5cm}|p{4cm}|c|}
        \hline
        \textbf{Test Case} & \textbf{Description} & \textbf{Expected Outcome} & \textbf{Status} \\
        \hline
        Insert internship position & Add a new internship listing & \ding{51} Successfully inserted & \ding{51} Pass \\
        \hline
        Invalid insert & Insert internship position with missing fields & \ding{55} Raises exception & \ding{51} Pass \\
        \hline
        Retrieve by ID & Fetch internship position details & \ding{51} Retrieves correct data & \ding{51} Pass \\
        \hline
        Retrieve by company ID & Fetch all internship positions for a company & \ding{51} Retrieves correct data & \ding{51} Pass \\
        \hline
        Retrieve by program name & Fetch internship positions by program name & \ding{51} Retrieves correct data & \ding{51} Pass \\
        \hline
        Search with filters & Filter internships by role, location, stipend & \ding{51} Retrieves filtered results & \ding{51} Pass \\
        \hline
        Update internship position status & Modify internship position state (e.g., Open, Closed) & \ding{51} Updates successfully & \ding{51} Pass \\
        \hline
    \end{tabular}
    \caption{Internship Position Database Test Cases}
    \label{tab:internship_position_database_tests}
\end{table}

\newpage
\subsubsection{Student Database Tests}

\begin{table}[h]
    \centering
    \renewcommand{\arraystretch}{1.3}
    \begin{tabular}{|p{5cm}|p{5cm}|p{4cm}|c|}
        \hline
        \textbf{Test Case} & \textbf{Description} & \textbf{Expected Outcome} & \textbf{Status} \\
        \hline
        Insert student & Add a new student record & \ding{51} Successfully inserted & \ding{51} Pass \\
        \hline
        Retrieve by ID & Fetch student details & \ding{51} Retrieves correct data & \ding{51} Pass \\
        \hline
        Retrieve by email & Fetch student using email & \ding{51} Retrieves correct data & \ding{51} Pass \\
        \hline
        Update student profile & Modify student information & \ding{51} Updates successfully & \ding{51} Pass \\
        \hline
    \end{tabular}
    \caption{Student Database Test Cases}
    \label{tab:student_database_tests}
\end{table}

\newpage
\subsubsection{University Database Tests}

\begin{table}[h]
    \centering
    \renewcommand{\arraystretch}{1.3}
    \begin{tabular}{|p{5cm}|p{5cm}|p{4cm}|c|}
        \hline
        \textbf{Test Case} & \textbf{Description} & \textbf{Expected Outcome} & \textbf{Status} \\
        \hline
        Insert university & Add a new university record & \ding{51} Successfully inserted & \ding{51} Pass \\
        \hline
        Retrieve by ID & Fetch university details & \ding{51} Retrieves correct data & \ding{51} Pass \\
        \hline
        Retrieve by email & Fetch university using email & \ding{51} Retrieves correct data & \ding{51} Pass \\
        \hline
        Update university details & Modify university profile information & \ding{51} Updates successfully & \ding{51} Pass \\
        \hline
    \end{tabular}
    \caption{University Database Test Cases}
    \label{tab:university_database_tests}
\end{table}

\newpage
\subsubsection{User Database Tests}

\begin{table}[h]
    \centering
    \renewcommand{\arraystretch}{1.3}
    \begin{tabular}{|p{5cm}|p{5cm}|p{4cm}|c|}
        \hline
        \textbf{Test Case} & \textbf{Description} & \textbf{Expected Outcome} & \textbf{Status} \\
        \hline
        Insert user & Add a new user record & \ding{51} Successfully inserted & \ding{51} Pass \\
        \hline
        Duplicate user insert & Attempt to insert duplicate email & \ding{55} Raises exception & \ding{51} Pass \\
        \hline
        Check email uniqueness & Verify if an email is unique before registration & \ding{51} Returns correct boolean & \ding{51} Pass \\
        \hline
        Retrieve user type by email & Fetch account type of a user & \ding{51} Retrieves correct type & \ding{51} Pass \\
        \hline
        Retrieve user type by ID & Fetch account type using user ID & \ding{51} Retrieves correct type & \ding{51} Pass \\
        \hline
    \end{tabular}
    \caption{User Database Test Cases}
    \label{tab:user_database_tests}
\end{table}

\newpage
\subsection{Model Tests}

The following tests verify the model functionality within the \textbf{Students \& Companies (S\&C) platform}. These tests check class attributes, getters, setters, conversions to dictionary format, and integration with the database layer.

\subsubsection{Assessment Model Tests}

\begin{table}[h]
    \centering
    \renewcommand{\arraystretch}{1.3}
    \begin{tabular}{|p{5cm}|p{5cm}|p{4cm}|c|}
        \hline
        \textbf{Test Case} & \textbf{Description} & \textbf{Expected Outcome} & \textbf{Status} \\
        \hline
        Get assessment ID & Retrieve the assessment ID & \ding{51} Returns correct ID & \ding{51} Pass \\
        \hline
        Get application ID & Retrieve associated application ID & \ding{51} Returns correct application ID & \ding{51} Pass \\
        \hline
        Get date & Retrieve assessment date & \ding{51} Returns correct date & \ding{51} Pass \\
        \hline
        Get link & Retrieve assessment link & \ding{51} Returns correct link & \ding{51} Pass \\
        \hline
        Convert to dictionary & Convert assessment object to dictionary format & \ding{51} Returns correct dict & \ding{51} Pass \\
        \hline
        Add assessment (valid) & Insert a new assessment & \ding{51} Successfully inserted & \ding{51} Pass \\
        \hline
        Add assessment (invalid) & Insert an assessment with error & \ding{55} Returns None & \ding{51} Pass \\
        \hline
        Get last assessment by application ID (valid) & Retrieve last assessment for a valid application & \ding{51} Returns correct assessment & \ding{51} Pass \\
        \hline
        Get last assessment by application ID (invalid) & Retrieve last assessment for a non-existent application & \ding{55} Returns None & \ding{51} Pass \\
        \hline
        Handle exception in get last assessment & Simulate an exception scenario & \ding{55} Raises exception & \ding{51} Pass \\
        \hline
    \end{tabular}
    \caption{Assessment Model Test Cases}
    \label{tab:assessment_model_tests}
\end{table}

\newpage
\subsubsection{Application Model Tests}

\begin{table}[h]
    \centering
    \renewcommand{\arraystretch}{1.3}
    \begin{tabular}{|p{5cm}|p{5cm}|p{4cm}|c|}
        \hline
        \textbf{Test Case} & \textbf{Description} & \textbf{Expected Outcome} & \textbf{Status} \\
        \hline
        Get application ID & Retrieve application ID & \ding{51} Returns correct ID & \ding{51} Pass \\
        \hline
        Get student ID & Retrieve associated student ID & \ding{51} Returns correct ID & \ding{51} Pass \\
        \hline
        Get internship position ID & Retrieve associated internship position ID & \ding{51} Returns correct ID & \ding{51} Pass \\
        \hline
        Get status & Retrieve application status & \ding{51} Returns correct status & \ding{51} Pass \\
        \hline
        Convert to dictionary & Convert application object to dictionary format & \ding{51} Returns correct dict & \ding{51} Pass \\
        \hline
        Add application (valid) & Insert a new application & \ding{51} Successfully inserted & \ding{51} Pass \\
        \hline
        Add application (invalid) & Insert application with missing fields & \ding{55} Returns None & \ding{51} Pass \\
        \hline
        Retrieve by ID (valid) & Fetch application by valid ID & \ding{51} Returns correct data & \ding{51} Pass \\
        \hline
        Retrieve by ID (invalid) & Fetch application by non-existent ID & \ding{55} Returns None & \ding{51} Pass \\
        \hline
        Handle exception in get by ID & Simulate an exception in retrieval & \ding{55} Raises exception & \ding{51} Pass \\
        \hline
    \end{tabular}
    \caption{Application Model Test Cases}
    \label{tab:application_model_tests}
\end{table}

\newpage
\subsubsection{University Model Tests}

\begin{table}[h]
    \centering
    \renewcommand{\arraystretch}{1.3}
    \begin{tabular}{|p{5cm}|p{5cm}|p{4cm}|c|}
        \hline
        \textbf{Test Case} & \textbf{Description} & \textbf{Expected Outcome} & \textbf{Status} \\
        \hline
        Get university ID & Retrieve the university ID & \ding{51} Returns correct ID & \ding{51} Pass \\
        \hline
        Get email & Retrieve email of the university & \ding{51} Returns correct email & \ding{51} Pass \\
        \hline
        Get university name & Retrieve the name of the university & \ding{51} Returns correct name & \ding{51} Pass \\
        \hline
        Convert to dictionary & Convert university object to dictionary format & \ding{51} Returns correct dict & \ding{51} Pass \\
        \hline
        Add university (valid) & Insert a new university & \ding{51} Successfully inserted & \ding{51} Pass \\
        \hline
        Add university (invalid) & Insert a university with missing fields & \ding{55} Returns None & \ding{51} Pass \\
        \hline
    \end{tabular}
    \caption{University Model Test Cases}
    \label{tab:university_model_tests}
\end{table}

\newpage
\subsubsection{User Model Tests}

\begin{table}[h]
    \centering
    \renewcommand{\arraystretch}{1.3}
    \begin{tabular}{|p{5cm}|p{5cm}|p{4cm}|c|}
        \hline
        \textbf{Test Case} & \textbf{Description} & \textbf{Expected Outcome} & \textbf{Status} \\
        \hline
        Get user ID & Retrieve user ID & \ding{51} Returns correct ID & \ding{51} Pass \\
        \hline
        Get email & Retrieve user email & \ding{51} Returns correct email & \ding{51} Pass \\
        \hline
        Get user type & Retrieve user type (e.g., student, company) & \ding{51} Returns correct type & \ding{51} Pass \\
        \hline
        Validate correct password & Check correct password validation & \ding{51} Returns True & \ding{51} Pass \\
        \hline
        Validate incorrect password & Check incorrect password validation & \ding{55} Returns False & \ding{51} Pass \\
        \hline
        Get user type by email (valid) & Retrieve user type using email & \ding{51} Returns correct type & \ding{51} Pass \\
        \hline
    \end{tabular}
    \caption{User Model Test Cases}
    \label{tab:user_model_tests}
\end{table}

\subsection{Authentication Service Tests}

These tests validate authentication-related functionalities, including validation of different input types, email uniqueness, and authentication attributes.

\begin{longtable}{|p{5cm}|p{6cm}|p{3.5cm}|c|}
    \hline
    \textbf{Test Case} & \textbf{Description} & \textbf{Expected Outcome} & \textbf{Status} \\
    \hline
    \endfirsthead
    
    \hline
    \textbf{Test Case} & \textbf{Description} & \textbf{Expected Outcome} & \textbf{Status} \\
    \hline
    \endhead
    
    \hline
    \multicolumn{4}{|r|}{{Continued on next page}} \\
    \hline
    \endfoot
    
    \hline
    \endlastfoot

    String validation (valid) & Check if a valid string is detected correctly & \ding{51} Returns True & \ding{51} Pass \\
    \hline
    String validation (invalid) & Check if an integer is incorrectly treated as a string & \ding{55} Returns False & \ding{51} Pass \\
    \hline
    Integer validation (valid) & Validate correct integer inputs & \ding{51} Returns True & \ding{51} Pass \\
    \hline
    Integer validation (invalid) & Validate incorrect non-integer inputs & \ding{55} Returns False & \ding{51} Pass \\
    \hline
    Float validation (valid) & Validate correct float inputs & \ding{51} Returns True & \ding{51} Pass \\
    \hline
    Float validation (invalid) & Validate incorrect non-float inputs & \ding{55} Returns False & \ding{51} Pass \\
    \hline
    Email uniqueness check (valid) & Check if an unused email is detected correctly & \ding{51} Returns True & \ding{51} Pass \\
    \hline
    Email uniqueness check (invalid) & Check if an existing email is detected correctly & \ding{55} Returns False & \ding{51} Pass \\
    \hline
    Email uniqueness check (exception) & Handle exceptions during uniqueness check & \ding{55} Raises exception & \ding{51} Pass \\
    \hline
    Email format validation (valid) & Validate correct email formats & \ding{51} Returns True & \ding{51} Pass \\
    \hline
    Email format validation (invalid) & Validate incorrect email formats & \ding{55} Returns False & \ding{51} Pass \\
    \hline
    Phone number validation (valid) & Validate correct phone number formats & \ding{51} Returns True & \ding{51} Pass \\
    \hline
    Phone number validation (invalid) & Validate incorrect phone number formats & \ding{55} Returns False & \ding{51} Pass \\
    \hline
    Password validation (valid) & Validate strong passwords & \ding{51} Returns True & \ding{51} Pass \\
    \hline
    Password validation (invalid) & Validate weak passwords & \ding{55} Returns False & \ding{51} Pass \\
    \hline
    URL validation (valid) & Validate correct URLs & \ding{51} Returns True & \ding{51} Pass \\
    \hline
    URL validation (invalid) & Validate incorrect URLs & \ding{55} Returns False & \ding{51} Pass \\
    \hline
    Location validation (valid) & Validate correct locations & \ding{51} Returns True & \ding{51} Pass \\
    \hline
    Location validation (invalid) & Validate incorrect locations & \ding{55} Returns False & \ding{51} Pass \\
    \hline
    Name validation (valid) & Validate correct names & \ding{51} Returns True & \ding{51} Pass \\
    \hline
    Name validation (invalid) & Validate incorrect names & \ding{55} Returns False & \ding{51} Pass \\
    \hline
    Degree program validation (valid) & Validate correct degree programs & \ding{51} Returns True & \ding{51} Pass \\
    \hline
    Degree program validation (invalid) & Validate incorrect degree programs & \ding{55} Returns False & \ding{51} Pass \\
    \hline
    GPA validation (valid) & Validate correct GPA values & \ding{51} Returns True & \ding{51} Pass \\
    \hline
    GPA validation (invalid) & Validate incorrect GPA values & \ding{55} Returns False & \ding{51} Pass \\
    \hline
    Graduation year validation (valid) & Validate correct graduation years & \ding{51} Returns True & \ding{51} Pass \\
    \hline
    Graduation year validation (invalid) & Validate incorrect graduation years & \ding{55} Returns False & \ding{51} Pass \\
    \hline
    Path validation (valid) & Validate correct paths & \ding{51} Returns True & \ding{51} Pass \\
    \hline
    Path validation (invalid) & Validate incorrect paths & \ding{55} Returns False & \ding{51} Pass \\
    \hline
    Optional path validation (valid) & Validate correct optional paths & \ding{51} Returns True & \ding{51} Pass \\
    \hline
    Optional path validation (invalid) & Validate incorrect optional paths & \ding{55} Returns False & \ding{51} Pass \\
    \hline
\end{longtable}
\caption{Authentication Service Test Cases}
\label{tab:authentication_service_tests}

\section{Additional Notes}

The code is well-structured, with thorough validation tests covering various input scenarios. It ensures proper handling of valid and invalid data types across different authentication and validation functions.

The test suite effectively verifies different edge cases, exceptions, and expected behaviors, demonstrating a responsible and systematic approach to software quality assurance.

The immediate solution to any encountered issues, such as downloading individual data, highlights the team's problem-solving skills and commitment to collaboration.

The tested group was highly responsive, addressing queries with prompt and detailed explanations, showing a strong understanding of their system.

\section{Areas for Improvement}

\subsection{Meta-Testing (Tests for Tests)}
While the current test suite is comprehensive, additional meta-tests could be implemented to verify the effectiveness and completeness of the test cases themselves. This would help ensure that:
\begin{itemize}
    \item Test functions actually cover all possible edge cases.
    \item Expected failures occur where intended.
    \item Mocks and patches behave correctly.
\end{itemize}

Implementing **meta-testing** would add an extra layer of validation to the test framework, reinforcing the reliability and robustness of the software testing process.


     \chapter{Effort Spent}
    \label{ch:effort_spents}%
    \begin{table}[H]
    \centering
    \begin{tabular}{|p{0.25\textwidth}|p{0.5\textwidth}|c|}
        \hline
        \textbf{Team Member} & \textbf{Task} & \textbf{Hours Spent} \\ 
        \hline
        Shreesh Kumar Jha & 
        \begin{enumerate}
            \item Introduction
            \item Architectural Design
            \item User Interface Design
            \item Requirements Traceability
            \item Implementation, Integration, and Test Plan
        \end{enumerate} & 35 \\ 
        \hline
        Samarth Bhatia & 
        \begin{enumerate}
            \item Introduction
            \item Architectural Design
            \item User Interface Design
            \item Requirements Traceability
            \item Implementation, Integration, and Test Plan
        \end{enumerate} & 37 \\ 
        \hline
        Satvik Sharma & 
        \begin{enumerate}
            \item Requirements Analysis
            \item UML Diagrams
            \item Backend Setup
            \item Backend Implementation
            \item API Refactor
        \end{enumerate} & 32 \\ 
        \hline
    \end{tabular}
    \caption{Effort spent by each member of the group.}
    \label{tab:effort_spent}
\end{table}


    \chapter{References}
    \label{ch:references}%
    \section{References}
\label{sec:references}%

\begin{itemize}
    \item Software Engineering 2 Course Materials, A.Y. 2024-2025.
    \item Daniel Jackson, \textit{Software Abstractions: Logic, Language, and Analysis}.
    \item Assignment RDD AY 2024-2025.pdf.
    \item  MongoDB, Inc. (n.d.). \href{https://www.mongodb.com/docs/manual/}{MongoDB Manual}.
    \item Node.js Foundation. (n.d.). \href{https://nodejs.org/en/docs/}{Node.js Documentation.}

\end{itemize}


% LIST OF TABLES
    \listoftables
    \cleardoublepage


\end{document}
