\section{Overview}
\label{sec:overview}
We adhered to the installation instructions found in \textbf{IT Document (ITD)v1}. The installation procedure was simple because of the comprehensive and well-organized guide. We encountered no problems throughout the installation, in contrast to other intricate setups. Every stage went without a hitch, and the system was successfully and consistently deployed.

\section{Dependency Installation}
\subsection{Backend Dependencies:}
To install the backend, we followed these steps:

\subsubsection{Installed Python and Virtual Environment}
\begin{itemize}
    \item Created a virtual environment and activated it using:
\end{itemize}


\begin{verbatim}
python3 -m venv env
source env/bin/activate  # For Linux/Mac  
env\Scripts\activate  # For Windows  
\end{verbatim}

\subsubsection{Installed Required Python Packages}
\begin{itemize}
    \item Using the \texttt{requirements.txt} file from the repository:
\end{itemize}

\begin{verbatim}
pip install -r requirements.txt
\end{verbatim}

\subsubsection{Installed Node.js and npm (For API Calls Testing)}
\begin{itemize}
    \item Installed Node.js (latest LTS version) and verified npm was installed properly:
\end{itemize}

\begin{verbatim}
node -v
npm -v
\end{verbatim}

\subsubsection{Installed MongoDB as the Database}
\begin{itemize}
    \item Since the backend uses MongoDB, we installed and ran it using:
\end{itemize}

\begin{verbatim}
mongod --dbpath=/data/db
\end{verbatim}

\subsection{Frontend Dependencies}

The frontend is built using React. The following dependencies were installed:

\subsubsection{Installed React and Required Libraries}
\begin{verbatim}
npm install
\end{verbatim}

\subsubsection{Checked TailwindCSS and Other UI Libraries}
\begin{itemize}
    \item Verified TailwindCSS and Flowbite were installed correctly.
\end{itemize}

\subsection{Backend Dependencies}

To install the backend, we followed these steps:

\subsubsection{Installed Python and Virtual Environment}
\begin{itemize}
    \item Created a virtual environment and activated it using:
\end{itemize}

\begin{verbatim}
python3 -m venv env
source env/bin/activate  # For Linux/Mac  
env\Scripts\activate  # For Windows  
\end{verbatim}

\subsubsection{Installed Required Python Packages}
\begin{itemize}
    \item Using the \texttt{requirements.txt} file from the repository:
\end{itemize}

\begin{verbatim}
pip install -r requirements.txt
\end{verbatim}

\subsubsection{Installed Node.js and npm (For API Calls Testing)}
\begin{itemize}
    \item Installed Node.js (latest LTS version) and verified npm was installed properly:
\end{itemize}

\begin{verbatim}
node -v
npm -v
\end{verbatim}

\subsubsection{Installed MongoDB as the Database}
\begin{itemize}
    \item Since the backend uses MongoDB, we installed and ran it using:
\end{itemize}

\begin{verbatim}
mongod --dbpath=/data/db
\end{verbatim}

\subsection{Frontend Dependencies}

The frontend is built using React. The following dependencies were installed:

\subsubsection*{Installed React and Required Libraries}
\begin{verbatim}
npm install
\end{verbatim}

\subsubsection*{Checked TailwindCSS and Other UI Libraries}
\begin{itemize}
    \item Verified TailwindCSS and Flowbite were installed correctly.
\end{itemize}

\section{Backend Installation}

\subsection{Cloned the Repository}
\begin{verbatim}
git clone https://github.com/edogriba/GribaldoRosa.git
cd ITD/backend
\end{verbatim}

\subsection{Started the Backend Server}
Activated the virtual environment and ran the server:
\begin{verbatim}
source env/bin/activate  # For Linux/Mac
python3 run.py
\end{verbatim}

\subsection{Database Initialization}
Ran the \texttt{createDB.py} script to set up MongoDB collections.

\section{Frontend Installation}

\subsection{Navigated to the Frontend Directory}
\begin{verbatim}
cd ../frontend
\end{verbatim}

\subsection{Installed Frontend Dependencies}
\begin{verbatim}
npm install
\end{verbatim}

\subsection{Ran the Frontend Application}
\begin{verbatim}
npm start
\end{verbatim}

\subsection*{Accessed the Application on Localhost}
Opened \texttt{http://localhost:3000} in a browser.

\section*{Testing Environment Setup}

\subsection{Tested API Endpoints using Postman}
\begin{itemize}
    \item All API requests responded as expected.
\end{itemize}

\subsection*{Tested User Registration \& Login}
\begin{itemize}
    \item Successfully created and logged in as a Student, Recruiter, and Admin.
\end{itemize}

\subsection{Internship Search \& Applications}
\begin{itemize}
    \item Verified that students could search and apply for internships.
\end{itemize}

\subsection{Messaging \& Complaints System}
\begin{itemize}
    \item Messages and complaints were successfully created and retrieved.
\end{itemize}

\section{Problems Encountered \& Documentation Incoherences}
\begin{itemize}
    \item \textbf{No Issues Faced During Installation.}
    \item The provided IT Document (ITD) was clear and accurate, ensuring a smooth setup.
    \item There were no missing dependencies or undocumented configurations.
\end{itemize}

\section{Observations}

\begin{itemize}
    \item Well-structured documentation made installation easy.
    \item Database initialization was straightforward without requiring manual data imports.
    \item No additional configurations were needed beyond those in the ITD.
\end{itemize}
