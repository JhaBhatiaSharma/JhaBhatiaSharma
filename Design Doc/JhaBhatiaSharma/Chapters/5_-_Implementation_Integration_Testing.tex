\section{Overview and Implementation Plan}
\label{sec:overview_implementation}

This chapter describes the InternHub – Students \& Companies (S\&C) platform's integration strategy, test plan, and implementation procedure. A systematic and effective development process will be ensured by using the \textbf{Bottom-Up approach}.

The implementation will start with basic, independent modules that do not need additional modules to work. Drivers for testing each module separately will be created. Modules will gradually be added to the system, taking the place of their associated drivers as they are implemented and tested. For further testing, each integrated module will need its own driver.

Before a system is fully integrated, smaller functional subsystems can be created using the Bottom-Up approach. With this incremental approach:
\begin{itemize}
    \item Testing is performed on smaller parts of the system initially and continues for each module as it becomes ready, making debugging and error tracking easier.
    \item Parallel development is facilitated, allowing separate teams to work on various elements simultaneously.
\end{itemize}

\section{Features Identification}
\label{sec:features_identification}

The features of the platform are prioritized based on their dependencies and importance, as outlined below:

\subsection{[F1] Login and Registration Features}
\begin{itemize}
    \item These are the core features required for students, companies, and administrators to access the platform.
    \item Include user registration, login, and secure authentication.
    \item As foundational features, they will be implemented first to support the functioning of subsequent features.
\end{itemize}

\subsection{[F2] Profile Management Features}
\begin{itemize}
    \item This set of features includes creating and managing profiles for students, companies, and administrators.
    \item Students can update personal details, upload CVs, and showcase skills.
    \item Companies can maintain organization profiles.
    \item These features serve as the foundation for the search and application functionalities.
\end{itemize}

\subsection{[F3] Internship Management Features}
\begin{itemize}
    \item Includes the ability for companies to create, update, and delete internship postings.
    \item Involves managing applications received for these internships.
    \item Requires proper implementation of profile management ([F2]).
    \item Will be developed subsequently after [F1] and [F2].
\end{itemize}

\subsection{[F4] Search and Filter Features}
\begin{itemize}
    \item Includes advanced search and filter functionalities:
    \begin{itemize}
        \item Students can find relevant internships.
        \item Companies can search through applicants.
    \end{itemize}
    \item These features depend on the successful implementation of profile and internship management ([F2] and [F3]).
\end{itemize}

\subsection{[F5] Application and Interview Features}
\begin{itemize}
    \item Includes submitting applications, reviewing them, scheduling interviews, and notifying users about interview updates.
    \item These features depend on internship management ([F3]) and profile management ([F2]).
\end{itemize}

\subsection{[F6] Complaint Handling Features}
\begin{itemize}
    \item Allows students and companies to lodge complaints and administrators to review and resolve them.
    \item Ensures user satisfaction and platform reliability.
    \item Relies on the proper implementation of profile management ([F2]) and dashboard functionalities ([F8]).
\end{itemize}

\subsection{[F7] Notification Features}
\begin{itemize}
    \item Ensures that students, companies, and administrators are notified about critical events:
    \begin{itemize}
        \item Interview schedules.
        \item Application updates.
        \item Complaint resolutions.
    \end{itemize}
    \item These features will be developed last as they depend on the correct functioning of all other features.
\end{itemize}

\subsection{[F8] Dashboard Features}
\begin{itemize}
    \item The dashboard provides an overview of active internships, applications, interviews, and platform activities for students, companies, and administrators.
    \item Consolidates data from various modules.
    \item Critical for system usability.
\end{itemize}

\subsection{Development Dependencies}
\label{subsec:development_dependencies}

The dependencies between the features ensure a structured and incremental implementation:
\begin{enumerate}
    \item [F1] Login and Registration Features serve as the foundation for all other functionalities.
    \item [F2] Profile Management Features are prerequisites for Internship Management ([F3]) and Search and Filter ([F4]).
    \item [F3] Internship Management Features depend on Profile Management ([F2]) and support Application and Interview Features ([F5]).
    \item [F4] Search and Filter Features depend on both Profile Management ([F2]) and Internship Management ([F3]).
    \item [F5] Application and Interview Features depend on Profile Management ([F2]), Internship Management ([F3]), and Search and Filter ([F4]).
    \item [F6] Complaint Handling Features rely on Profile Management ([F2]) and Dashboard Features ([F8]).
    \item [F7] Notification Features depend on the correct functioning of all other features.
    \item [F8] Dashboard Features consolidate data from all modules and rely on their successful implementation.
\end{enumerate}

This dependency-based plan ensures that features are developed systematically, reducing errors and facilitating incremental testing.

\section{Implementation Strategy}
\label{subsec:implementation_strategy}

\subsection{Overview and Integration Plan}
\label{subsubsec:integration_plan}

A systematic \textbf{bottom-up method} is used to integrate the system's components. The emphasis will be on developing solid foundational modules that can be gradually combined, starting with the essential elements. Before each module is integrated into the larger system, it will be tested using the relevant drivers. This approach enables:
\begin{itemize}
    \item Parallel programming.
    \item Effective debugging.
    \item Incremental functional validation.
\end{itemize}

The following crucial areas will be included in the integration process:

\paragraph{Core Model Integration}

The core model integration serves as the system's cornerstone, incorporating the data models necessary for the platform's operation. These include:
\begin{itemize}
    \item \textbf{User Model:} Manages user-related information.
    \item \textbf{Resume Model:} Handles CVs and profile details.
    \item \textbf{Internship Model:} Stores internship data.
    \item \textbf{Application Model:} Tracks internship applications.
    \item \textbf{Interview Model:} Manages interview scheduling and feedback.
    \item \textbf{Complaint Model:} Logs and tracks complaints.
\end{itemize}
\begin{figure}[H]
    \begin{center}
        \includegraphics[width=0.79\linewidth]{JhaBhatiaSharma/imagesDD/CoreModelIntegration.png}
        \caption{Core Model Integration}
        \label{fig:coreModelIntegration}
    \end{center}
\end{figure}
Each model will communicate with the database layer to perform CRUD operations and ensure data consistency and integrity. A \textbf{Model Driver} will be implemented to test these models individually and verify their efficient interaction with the database layer. Once validated, these models will be integrated into their corresponding manager components.

\paragraph{Authentication Integration}

Authentication integration focuses on user registration and login procedures. The \textbf{Authentication Manager} will handle interactions between the data models and the user interfaces for registration and login. Key functionalities include:
\begin{itemize}
    \item Secure communication with the Registration System.
    \item Data validation.
    \item Access token generation.
    \item Credential validation workflows.
\end{itemize}
\begin{figure}[H]
    \begin{center} \includegraphics[width=0.18\linewidth]{JhaBhatiaSharma/imagesDD/AuthenticationIntegration.png}
    \caption{Authentication Integration}
        \label{fig:authenticationIntegration}
    \end{center}
\end{figure}

A driver will test the authentication workflows, ensuring stability before integrating the Authentication Manager with the larger system to facilitate secure user authentication and seamless login.

\paragraph{Profile Management Integration}
The \textbf{Profile Manager} handles the creation and management of profiles for administrators, companies, and students. This module will interact directly with the Authentication Manager to ensure only authorized users can manage profiles. Key features include:
\begin{itemize}
    \item Profile creation.
    \item Profile editing.
    \item Profile retrieval.
\end{itemize}
\begin{figure}[H]
    \begin{center}
        \includegraphics[width=0.79\linewidth]{JhaBhatiaSharma/imagesDD/ProfileIntegration.png}
        \caption{Profile Management Integration}
        \label{fig:profileManagement}
    \end{center}
\end{figure}
The \textbf{Profile Driver} will test these features to validate functionality. This integration is critical for enabling personalized user experiences and creating user identification across the platform.

\paragraph{Internship Management Integration}
The \textbf{Internship Manager} is central to coordinating the posting and management of internship opportunities. It integrates with:
\begin{itemize}
    \item The Profile Manager to verify recruiter responsibilities and permissions.
    \item The Application System to manage student applications and facilitate filtering and internship searches.
\end{itemize}
\begin{figure}[H]
    \begin{center}
        \includegraphics[width=0.79\linewidth]{JhaBhatiaSharma/imagesDD/InternshipManagementIntegration.png}
        \caption{Internship Management Integration}
        \label{fig:internshipManagement}
    \end{center}
\end{figure}
A driver will test functionalities such as posting, deleting, and searching for internships to ensure safe and effective internship management.

\paragraph{Interview and Communication Integration}
The \textbf{Interview Manager} and \textbf{Communication Manager} are responsible for interview scheduling and stakeholder communication. The integration includes:
\begin{itemize}
    \item The Scheduling System to handle interview setups.
    \item The Feedback System to collect post-interview feedback.
\end{itemize}
\begin{figure}[H]
    \begin{center}
        \includegraphics[width=0.59\linewidth]{JhaBhatiaSharma/imagesDD/InterviewandCommunicationIntegration.png}
        \caption{Internship and Communication Integration}
        \label{fig:internshipManagement}
    \end{center}
\end{figure}
The Communication Manager ensures smooth communication and notifications. A \textbf{Communication Driver} will test information flow between the user interface and interview management systems.

\paragraph{Administrative Features Integration}
The \textbf{Admin Manager} is integrated with the following systems:
\begin{itemize}
    \item Monitoring System.
    \item Reporting System.
    \item Complaint System.
\end{itemize}
\begin{figure}[H]
    \begin{center}
      \includegraphics[width=0.59\linewidth]{JhaBhatiaSharma/imagesDD/AdministrativeFeaturesIntegration.png}
        \caption{Administrative Features Integration}
        \label{fig:adminstrativefeatures}
    \end{center}
\end{figure}
This integration includes the Communication Manager to facilitate administrative communication. Functionalities such as complaint management, platform activity monitoring, and report generation will be tested using the \textbf{Admin Driver}. This ensures stability and efficiency in administrative functions.

\paragraph{Final System Integration}
In the final integration phase, all primary managers—Authentication Manager, Profile Manager, Internship Manager, Interview Manager, Communication Manager, and Admin Manager—will be connected to the \textbf{Dashboard Manager}. This integration ensures:
\begin{itemize}
    \item Seamless coordination of data and processes across the platform.
    \item A unified user experience.
\end{itemize}
\begin{figure}[H]
    \begin{center}
      \includegraphics[width=0.18\linewidth]{JhaBhatiaSharma/imagesDD/FinalSystemIntegration.png}
        \caption{Final System Integration}
        \label{fig:finalsystemintegration}
    \end{center}
\end{figure}

The Dashboard Manager acts as the main interface, consolidating data and system processes. This phase marks the completion of the integration process, delivering a fully functional and user-friendly platform.

\section{System Testing Strategy}
\label{sec:system_testing_strategy}

Every newly created component will go through extensive testing before being incorporated into the system to guarantee the platform's accuracy and dependability. To verify each component's unique functionality, drivers will be utilized. After integration, a new driver will test the compatibility of the new component with the existing system, ensuring that module properties are maintained and the overall workflow remains unaffected. Once all components have been integrated, the entire system will undergo testing to confirm its proper operation and ensure the absence of flaws. The following testing methods will be employed:

\subsection{Functional Testing}
Functional testing will confirm that all objectives, specifications, and use cases are satisfied and that the platform complies with the capabilities described in the RASD document. This testing will simulate user scenarios and confirm the system's proper workflow by verifying expected results. Key elements that will be tested include:
\begin{itemize}
    \item Applications
    \item Internship postings
    \item Profile management
    \item Login
    \item Interview scheduling
    \item Complaint handling
\end{itemize}

\subsection{Load Testing}
The system's behavior under various workloads will be evaluated through load testing to identify:
\begin{itemize}
    \item Memory leaks
    \item Buffer overflows
    \item Inefficient memory management
\end{itemize}
This testing ensures that the platform can efficiently manage multiple requests simultaneously and remains stable during periods of high user activity.

\subsection{Performance Testing}
Performance testing will identify bottlenecks and assess how quickly the system responds to demanding workloads. This ensures that:
\begin{itemize}
    \item The platform supports many users concurrently with minimal latency.
    \item Optimization opportunities in the underlying algorithms are identified to improve overall system performance.
\end{itemize}

\subsection{Stress Testing}
Stress testing will simulate extreme scenarios such as:
\begin{itemize}
    \item A large number of users accessing the system simultaneously.
    \item A reduction in processing power.
\end{itemize}
This testing will confirm the platform's robustness and resilience, ensuring minimal user disruption during emergencies or unexpected peak loads.

\subsection{User Interface Testing}
User interface testing will validate the platform's usability and accessibility across a variety of devices and browsers. This testing will:
\begin{itemize}
    \item Ensure a smooth and uniform experience for all user types—students, companies, and administrators.
    \item Verify compatibility with various screen sizes and resolutions.
\end{itemize}

\subsection{Comprehensive Testing Approach}
The system's functionality, reliability, and scalability will be fully verified by combining the following testing techniques:
\begin{itemize}
    \item Functional Testing
    \item Load Testing
    \item Performance Testing
    \item Stress Testing
    \item User Interface Testing
\end{itemize}
This comprehensive approach guarantees:
\begin{itemize}
    \item A flawless user experience.
    \item Compliance with the specifications outlined in the RASD document.
\end{itemize}
